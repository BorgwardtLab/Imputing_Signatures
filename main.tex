\documentclass{article}

\usepackage{microtype}
\usepackage{graphicx}
\usepackage{booktabs} % for professional tables
\usepackage{caption}
\usepackage{subcaption} % for subcaption boxes
\usepackage{amsmath}
\usepackage{bm}      % for bold greek letters
\usepackage{amssymb} % more math symbols
\usepackage{amsthm}
\usepackage{bbm}  % blackboard 1
\usepackage{hyperref}
\usepackage{paralist}
\usepackage{wrapfig}
\newcommand{\theHalgorithm}{\arabic{algorithm}}
\usepackage{comment}
\usepackage{multicol} %multicol in figure
\PassOptionsToPackage{sort, numbers, compress}{natbib}

\usepackage{neurips_2020} %[preprint]
\setcitestyle{square}

\renewcommand{\subsubsection}[1]{\textbf{#1}

} % newlines are deliberate
\DeclareMathOperator*{\argmax}{arg\,max}
\DeclareMathOperator*{\argmin}{arg\,min}
\DeclareMathOperator*{\E}{\mathbb{E}}
\newcommand{\reals}{\mathbb{R}}
\newcommand{\naturals}{\mathbb{N}}
\newcommand{\sig}{\mathrm{Sig}^N}
\newcommand{\dataspace}{\mathcal{X}}
\newcommand{\lspace}{\mathcal{Y}}
\newcommand{\seriesspace}{\mathcal{S}}
\newtheorem{theorem}{Theorem}
\newcommand{\data}[1]{\mbox{\textsc{#1}}}   % can be used to typeset arbitrary data set set names

\title{Path Imputation Strategies for Signature Models of Irregular Time Series}
% Alternative Title:
%% Path Imputation Strategies with Signature Models for Irregularly Sampled Partially Observed Multivariate Time Series.

\author{
    Michael Moor
    \And
	Max Horn
	\And
	Christian Bock
	\And
	Bastian Rieck
	\And
	Karsten Borgwardt
	\AND \\[-12pt]
	Department of Biosystems Science and Engineering, ETH Zurich
	\\
	\hspace{-6.5em}\texttt{\{michael.moor, max.horn, christian.bock,}
	\\
	\hspace{6.5em}\texttt{bastian.rieck, karsten.borgwardt\}@\hspace{0.1pt}bsse.ethz.ch}
} % TODO: are the emails I've guessed correct?

\begin{document}
\maketitle

\begin{abstract}
The signature transform is a `universal nonlinearity' on the space of
continuous vector-valued paths, and has received attention for use in
machine learning. However, real-world temporal data is usually observed at discrete points in time, and must first be transformed into a continuous path before
signature techniques can be applied. We make this (typically implicit) step explicit by characterising it as an imputation
problem, and assess the impact of various imputation strategies when
applying signatures to irregular time series data. For one of these
strategies, Gaussian process~(GP) adapters, we propose an
extension~(GP-PoM) that makes uncertainty information directly available
to the subsequent classifier while at the same time preventing costly
Monte-Carlo~(MC) sampling. In our experiments, we find that the choice
of imputation  drastically affects shallow signature models, whereas
deeper architectures are more robust. Next, we observe that
uncertainty-aware \emph{predictions}~(based on GP-PoM or indicator
imputations) are beneficial for predictive performance, even compared to
the uncertainty-aware \emph{training} of conventional GP adapters.
%
% BR: sounds a little bit like the ending of the paper already; consider
% reformulating.
In conclusion, we have demonstrated that the path construction is indeed
crucial for signature models and that our proposed strategy leads to
competitive performance in general, while improving robustness of
signature models in particular.
\end{abstract} % TODO: touch this up a bit, but I think the essential idea is there.

%%%%%%%%%%%%%%%%%%%%%%%%%%%%%%%%%%%%%%%%%%%%%%%%%%%%%%%%%%%%%%%%%%%%%%%%
\section{Introduction}
%%%%%%%%%%%%%%%%%%%%%%%%%%%%%%%%%%%%%%%%%%%%%%%%%%%%%%%%%%%%%%%%%%%%%%%%

Originally described by \citet{Chen54, Chen57, Chen58} and popularised
in the theory of rough paths and controlled differential
equations~\cite{lyons1998differential, FritzVictoir10, lyons2014rough},
the \emph{signature transform}, also known as the \emph{path signature}
or simply \emph{signature}, acts on a continuous vector-valued path of
bounded variation, and returns a graded sequence of statistics, which
determine a path up to a negligible equivalence class. Moreover,
\emph{every} continuous function of a path can be recovered by applying
a linear transform to this collection of statistics~\citep[Proposition
A.6]{kidger2019deep}.
%
This `universal nonlinearity' property makes the signature a promising nonparametric
feature extractor with beneficial properties in both generative and
supervised learning scenarios.
%
Given their similarities, we may hope that tools that
apply to continuous paths can \emph{also} be extended to multivariate
time series. But since multivariate time series are not continuous
paths, one first needs to construct a continuous path to apply signature techniques.

Previous work~\citep{levin2013, kidger2019deep, fermanian2019embedding}
characterised this construction as an embedding problem, and
typically considered it a minor technical detail.
This is exacerbated by the---perfectly sensible---behaviour of software
for computing the signature~\citep{iisignature, signatory}, which
commonly considers a continuous piecewise linear path as an input,
described by its sequence of knots, i.e.\ values.
%
Since such sequences resemble a sequence of data, the signature is
sometimes interpreted as operating on sequences of data rather than on
paths~\cite{kidger2019deep, levin2013}.
%
By contrast, here we show that considering the path construction
process is crucial for achieving competitive predictive
performance: we reinterpret the task of constructing a continuous path,
turning it from an embedding problem to an imputation problem, which we
call \emph{path imputation}.
%

While previous research concerning the signature transform
focussed on its excellent theoretical properties, such as sampling
independence~\citep[Proposition A.7]{kidger2019deep}, our findings show
that this does not
necessarily correspond to empirical performance.
%
We perform a thorough investigation of multiple imputation schemes in combination with various models that can potentially employ signatures.
Furthermore, motivated by the fact that missingness itself can be
informative for time series classification~\citep{rubin1976inference}, we propose a novel extension to Gaussian process adapters~\citep{li2016scalable} which exploits uncertainty information during each prediction step, which we show is beneficial for signature models, but also of independent interest.
%
% We find that approaches
% that have access to uncertainty information \emph{during prediction}
% tend to outperform the competitors, even uncertainty-aware frameworks
% for which this information is only available during \emph{training},
% such as Gaussian Process~(GP) adapters~\citep{li2016scalable,
% futoma2017mgp}~(for which we propose a novel extension to incorporate
% uncertainty information, thus improving predictive performance).
%
% Throughout our experiments, we observe that \emph{deep signature models}~\citep{kidger2019deep} consistently
% outperform other signature models~(which in turn suffer from increased
% variance, thus reducing their efficacy).
%
We make our code anonymously available under \url{https://osf.io/ktc96/?view_only=62d41b4e60f64d49a3fb100a45d08116}.

% %%%%%%%%%%%%%%%%%%%%%%%%%%%%%%%%%%%%%%%%%%%%%%%%%%%%%%%%%%%%%%%%%%%%%%%%
% % Introduction OLD
% %%%%%%%%%%%%%%%%%%%%%%%%%%%%%%%%%%%%%%%%%%%%%%%%%%%%%%%%%%%%%%%%%%%%%%%%

% \section{Introduction}

% Originally described by \citet{Chen54, Chen57, Chen58} and popularised
% in the theory of rough paths and controlled differential
% equations~\cite{lyons1998differential, FritzVictoir10, lyons2014rough},
% the \emph{signature transform}, also known as the \emph{path signature}
% or simply \emph{signature}, acts on a continuous vector-valued path of
% bounded variation, and returns a graded sequence of statistics, which
% determine a path up to a negligible equivalence class. Moreover,
% it can be shown~\cite[Proposition A.6]{kidger2019deep} that \emph{every}
% continuous function of a path can be recovered by applying a linear
% transform to this collection of statistics. This `universal
% nonlinearity' property makes the signature a promising nonparametric
% feature extractor with beneficial properties in both generative and
% % TODO: consider adding more citations here
% supervised learning scenarios~\citep{kidger2019deep}.
% %
% Given their similarities, it is natural to hope that the tools that
% apply to continuous paths may \emph{also} be applied to multivariate
% time series. Of course, multivariate time series are not continuous
% paths---in fact, for signature techniques to be applied, one first needs
% to construct a continuous path from this kind of data.

% In previous work~\cite{kidger2019deep, levin2013, fermanian2019embedding}, this
% construction has been characterised as an embedding problem, and was
% typically glossed over as a rather unimportant detail.
% %
% This is exacerbated by the---perfectly sensible---behaviour of software
% for computing the signature \cite{iisignature, signatory}, which
% commonly considers a continuous piecewise linear path as an input,
% % TODO: is there a better definition of 'knot'? I am aware of the
% % meaning in terms of spline approximations, but maybe we could clarify
% % this here.
% described by its sequence of knots, i.e.\ values.
% %
% Since such sequences resemble a sequence of data, the signature is
% sometimes interpreted as operating on sequences of data rather than on
% paths~\cite{kidger2019deep, levin2013}.
% %
% We show that this view is somewhat limiting and that, in order to
% achieve competitive predictive performance for real-world time series
% classification, it is crucial to consider the path construction process.

% Specifically, in this paper we reinterpret the task of
% constructing a continuous path, by turning it from an embedding problem
% into an imputation problem. To distinguish this from traditional
% imputation problems, we refer to this as \emph{path imputation}.
% %
% While a large part of the previous literature on the signature transform
% focused on its excellent theoretical properties, in particular sampling
% independence~\cite[Proposition A.7]{kidger2019deep},~\cite{signatorydocumentation},~\cite[Section 2]{toth2019gp}, our findings show that this does not
% necessarily correspond to excellent empirical performance.
% %
% However, it has previously been established that both observation rates
% and missingness itself can carry information in time series
% classification tasks~\citep{rubin1976inference, gelman2007dataanalysis}.
% %
% Our reinterpreted view of path construction as an imputation problem
% permits us to identify that this performance gap is likely another
% manifestation of the phenomenon of glossing over imputation details.

% We therefore perform a thorough empirical investigation of various
% imputation schemes in combination with multiple models that can potentially
% employ signatures. We find that approaches that have access to
% uncertainty information \emph{during prediction} tend to outperform the
% competitors, even uncertainty-aware frameworks for which this
% information is only available during \emph{training}, such as Gaussian
% Process~(GP) adapters~\citep{li2016scalable, futoma2017mgp}.
% %
% Motivated by this, we propose a novel extension of GP adapters that
% directly incorporates uncertainty information into the prediction step,
% thus improving predictive performance.
% Throughout our experiments, we observe that the recently introduced
% \emph{deep signature models}~\citep{kidger2019deep} consistently
% outperform other signature models~(which in turn suffer from increased
% variance, thus reducing their efficay).
% %
% We make our code publicly available under XXXXXX.


%%%%%%%%%%%%%%%%%%%%%%%%%%%%%%%%%%%%%%%%%%%%%%%%%%%%%%%%%%%%%%%%%%%%%%%%
\section{Related work} %(NEW)
%%%%%%%%%%%%%%%%%%%%%%%%%%%%%%%%%%%%%%%%%%%%%%%%%%%%%%%%%%%%%%%%%%%%%%%%

A key motivation for this work is the use of the signature transform in
machine learning: recent work~\citep{primer2016,
kormilitzlin2016, yang2016rotation, li2017lpsnet, yang2017leveraging,
PerezArribas2018, morrill2019sepsis} typically employed the signature transform as a nonparametric feature extractor, on top of which
a model is learnt. A growing body of work has also investigated how to
integrate the signature transform more tightly with neural networks; 
\citet{jeremythesis}, \citet{logsigrnn}, and \citet{kidger2019deep} all
study how to use the signature transform\footnote{And the related
logsignature transform; the difference between them will not be
important for us.} within typical neural network models.
\citet{chevyrev2018signature, kiraly2019kernels} study how the signature
transform may be used to define a \emph{kernel}---i.e.\ a symmetric,
positive definite function that is typically used as a similarity
measure---on path space, while \citet{toth2019gp} show how this kernel
may be used to define a Gaussian process.
%
In much of this work, data has been converted into a continuous path via
linear interpolation. Some authors~\citep{primer2016,
fermanian2019embedding} have additionally considered `rectilinear'
interpolation, which is similar. \citet{levin2013} present the
`time-joined transformation', which is a hybrid of the two, such that
the resulting path exhibits a causal dependence on the data.  However,
to our knowledge, no prior work has regarded (and empirically investigated) this as an imputation
problem.

\paragraph{Imputation schemes} The general problem of imputing data is well-known and well-studied, and
we will not attempt to describe it here; see for example \citet[Chapter 25]{gelman2007dataanalysis}.
% Common simple imputation schemes are to impute missing values as the mean of its channel, or to forward fill the last observed value, although it has been observed that incautious use of such methods can result in heavily biased models \cite{Molnar2008}.
Imputation methods typically only fill in missing discrete data
points, and do not attempt to impute the underlying continuous path.
Gaussian process adapters~\citep{li2016scalable}, by contrast, are
capable of imputing a \emph{full} continuous path, from which we
may sample arbitrarily. Hence, this framework will be
considered more closely in this paper.
% These methods typically only fill in missing discrete data points, and do not attempt to impute the underlying continuous path.
% %
% One approach that does impute a \emph{full} continuous path is
% a Gaussian process adapter~\citep{li2016scalable, futoma2017mgp}. The
% data is modelled as coming from a Gaussian process. The posterior
% distribution, having observed the data, then gives continuous paths
% through data space. This then allows, for example, for uniformly
% resampling the data. Gaussian process adapters will be important to this
% study, although in our case they are of interest not because of uniform
% resampling, but because they provide a full continuous path, which we may
% sample arbitrarily.
We note that there are also other approaches that
perform continuous-time imputation~\citep{shukla2018interpolationprediction},
or methods that skip an explicit imputation step altogether to directly classify irregular time series~\citep{che2018recurrent, rubanova2019latent, kidger2020neuralcde}. However, for the scope of this work, i.e. assessing the impact of path imputations for the signature, we will not further investigate these approaches.

%%%%%%%%%%%%%%%%%%%%%%%%%%%%%%%%%%%%%%%%%%%%%%%%%%%%%%%%%%%%%%%%%%%%%%%%
\section{Background} %NEW
%%%%%%%%%%%%%%%%%%%%%%%%%%%%%%%%%%%%%%%%%%%%%%%%%%%%%%%%%%%%%%%%%%%%%%%%

Let $f = (f_1, \dots, f_d) \colon [a, b] \to \reals^d$ be a continuous, piecewise
differentiable path. Then the \emph{signature transform up to depth $N$}
is
% >> Single Column Equation:
\begin{equation}\label{eq:signature}
    \sig(f)=\left(\left(\underset{\,a<t_{1}<\cdots<t_{k}<b}{\int \cdots \int} \prod_{j=1}^{k} \frac{\mathrm{d} f_{i_{j}}}{\mathrm{d} t}\left(t_{j}\right) \mathrm{d} t_{1} \cdots \mathrm{d} t_{k}\right)_{1 \leq i_{1}, \ldots, i_{k} \leq d}\right)_{1 \leq k \leq N}.
\end{equation}

This definition can be extended to paths of bounded variation by
replacing these integrals with Stieltjes integrals with respect to each
$f_{i_j}$.
%
In brief, the signature transform may be interpreted as extracting
information about \emph{order} and \emph{area} of a path.
%
One may interpret its terms as `the area/order of one channel with
respect to some collection of other channels'.
For an exposition on the properties of the signature transform and its
use in machine learning, please refer to \citet{primer2016} or \citet[Appendix A]{kidger2019deep}.

%%%%%%%%%%%%%%%%%%%%%%%%%%%%%%%%%%%%%%%%%%%%%%%%%%%%%%%%%%%%%%%%%%%%%%%%
\paragraph{Notation}
%%%%%%%%%%%%%%%%%%%%%%%%%%%%%%%%%%%%%%%%%%%%%%%%%%%%%%%%%%%%%%%%%%%%%%%%
%
Given a set $A$, we define the space of time series over $A$ by
%
\begin{equation}
    \seriesspace(A) = \{((t_1, x_1), \ldots, (t_n, x_n)) \,\vert\, t_i \in \reals, x_i \in A, n \in \naturals, \text{ such that } t_1 \leq \cdots \leq t_n\}.\label{eq:seriesspace}
\end{equation}
%Two column version:
% \begin{equation}\label{eq:seriesspace}
%   \begin{split}
%     \seriesspace(A) = &\{((t_1, x_1), \ldots, (t_n, x_n)) \,\vert\, t_i \in \reals, x_i \in A,\\
%                       &\hspace{7em}n \in \naturals, \text{ s.t. } t_1 \leq \cdots \leq t_n\}.
%   \end{split}
% \end{equation}
%
Furthermore, let $\lspace$ be a set and let $\dataspace_j = \reals$ for $j \in \{1,
\ldots, d\}$ and $d \in \naturals$. Then we assume that we observe
a dataset of labelled time series $(\mathbf{x}_k, y_k)$ for $k \in \{1,
\ldots, N\}$, where $\mathbf{x}_k \in \seriesspace(\dataspace^*)$ and
$y_k \in \lspace$, with $\dataspace^* = \prod_{j = 1}^d(\dataspace_j
\cup \{*\})$ and $*$ representing no observation.
%
We similarly define $\dataspace = \prod_{j = 1}^d\dataspace_j$. Thus,
$\dataspace$ is the data space, while $\dataspace^*$ is the data space
allowing missing data, and $\lspace$ is the set of labels.

% \subsubsection{Terminology for imputation schemes}
% To avoid ambiguity, we will refer to standard imputation schemes, such as forward-fill, as a \emph{data-imputation} scheme, meaning to impute missing data points. In contrast we will use \emph{path-imputation} to describe the more general task of imputing the full continuous path.

%%%%%%%%%%%%%%%%%%%%%%%%%%%%%%%%%%%%%%%%%%%%%%%%%%%%%%%%%%%%%%%%%%%%%%%%
\paragraph{Gaussian process adapter} \label{sec: GPadapter}
%%%%%%%%%%%%%%%%%%%%%%%%%%%%%%%%%%%%%%%%%%%%%%%%%%%%%%%%%%%%%%%%%%%%%%%%
%
Some of the imputation schemes we consider are based on
the uncertainty aware-framework of multi-task Gaussian process
adapters~\citep{li2016scalable, futoma2017mgp}.
%
Let $\mathcal{W}, \mathcal{H}$ be some sets. Let $\ell \colon \lspace
\times \lspace \to [0, \infty)$ be a loss function. Let $F \colon
\dataspace^{[a, b]} \times \mathcal{W} \to \lspace$, be some (typically
neural network) model, with $\mathcal{W}$ interpreted as a space of
parameters. Let \begin{align*} \mu \colon [a, b] \times
\seriesspace(\dataspace^*) \times \mathcal{H} &\to \dataspace\\ \Sigma
\colon [a, b] \times [a, b] \times \seriesspace(\dataspace^*) \times
\mathcal{H} &\to \dataspace    \end{align*} be mean and covariance
functions, with $\mathcal{H}$ interpreted as a space of
hyperparameters. The dependence on $\seriesspace(\dataspace^*)$ is to
represent conditioning on observed values.

Then the goal is to solve
\begin{equation}\label{eq:gp-mc}
\argmin_{\mathbf{w} \in \mathcal{W},\bm{\eta} \in \mathcal{H}} \sum_{k=1}^N \overbrace{\rule{0pt}{0.5cm} \mathbb{E}_{\mathbf{z}_k \sim \mathcal{N}\left( \mu(\cdot, \mathbf{x}_k, \eta), \Sigma(\cdot, \cdot, \mathbf{x}_k, \eta)\right) } \big[ \ell(F(\mathbf{z}_k, \mathbf{w}),
y_k) \big] }^{\text{$E_k$}}.
\end{equation}
%
As this expectation is typically not tractable, it is estimated by Monte
Carlo (MC) sampling with $S$ samples, i.e.\
\begin{equation}
E_k \approx \frac{1}{S} \sum_{s=1}^{S} \ell(F(\mathbf{z}_{s, k}, \mathbf{w}), y_k),
\end{equation}
%
where
%
\begin{equation}
    \mathbf{z}_{s, k} \sim \mathcal{N}\left( \mu(\,\cdot\,, \mathbf{x}_k, \eta), \Sigma(\,\cdot\,, \,\cdot\,, \mathbf{x}_k, \eta)\right).
\end{equation}

Alternatively, one may forgo allowing the uncertainty to propagate through $F$ by instead passing the posterior mean directly to $F$; this corresponds to solving
\begin{equation}\label{eq:gp-mean}
\argmin_{\mathbf{w} \in \mathcal{W},\bm{\eta} \in \mathcal{H}} \sum_{k=1}^N \ell(F(\mu(\,\cdot\,,\mathbf{x}_k, \eta), \mathbf{w}), y_k).
\end{equation}

% %%%%%%%%%%%%%%%%%%%%%%%%%%%%%%%%%%%%%%%%%%%%%%%%%%%%%%%%%%%%%%%%%%%%%%%%
% \section{Background} %OLD
% %%%%%%%%%%%%%%%%%%%%%%%%%%%%%%%%%%%%%%%%%%%%%%%%%%%%%%%%%%%%%%%%%%%%%%%%

% % TODO: can we summarise what we are doing here? What are the most
% % important points of this section?
% %
% % Maybe: 'In the following, we provide a cursory introduction to the
% % path signature methods. Please refer to XXX for a more in-depth
% % description.'

% \subsection{Path signatures}

% Given a continuous, piecewise differentiable path $f \colon [a, b] \to
% \reals^d$, the \emph{signature transform up to depth $N$} may be defined
% by
% %
% \begin{equation}
%     \sig(f)=\left(\left(\underset{\,a<t_{1}<\cdots<t_{k}<b}{\int \cdots \int} \prod_{j=1}^{k} \frac{\mathrm{d} f_{i_{j}}}{\mathrm{d} t}\left(t_{j}\right) \mathrm{d} t_{1} \cdots \mathrm{d} t_{k}\right)_{1 \leq i_{1}, \ldots, i_{k} \leq d}\right)_{1 \leq k \leq N}.
% \end{equation}

% % \begin{wrapfigure}{l}{4cm}
% %     \centering
% %     \vspace{-2em}
% % 	\includegraphics[width=0.3\columnwidth]{figures/sig_path1.pdf}
% % 	\caption{Given a path, shown in bold, its Levy area is its signed area with respect to the chord joining its endpoints.}\label{fig:sig_path}
% % 	\vspace{-2em}
% % \end{wrapfigure}

% %This definition may be extended to paths of merely bounded variation by replacing these integrals with Riemann--Stieltjes integrals, and extended further to paths of bounded $p$-variation by replacing them with rough integrals \cite{lyons1998differential}.

% In brief, the signature transform may be interpreted as extracting
% information about \emph{order} and \emph{area} of a path.
% %
% One may interpret its terms as `the area/order of one channel with
% respect to some collection of other channels'.
% %
% % As an explicit example of the sort of information extracted, one of the things computed by the signature is the \emph{Levy area} of the path, shown for a one-dimensional example in Figure \ref{fig:sig_path}.
% %
% For an exposition on the properties of the signature transform and its
% use in machine learning, please refer to \citet{primer2016} or \citet[Appendix A]{kidger2019deep}.
% %
% %There is one fact that will prove important to our arguments later, however, so for completeness we repeat it here:
% %\begin{theorem}[Invariance to reparameterisation]\label{theorem:invariancetime}
% %Let $f \colon [a, b] \to \reals^d$ be a continuous piecewise differentiable path. Let $\psi \colon [a, b] \to [c, d]$ be continuously differentiable, increasing, and surjective. Then $\sig(f) = \sig(f \circ \psi)$.
% %\end{theorem}

% %\subsubsection{Comparison to the Fourier and wavelet transforms}
% %The signature transform exhibits a certain similarity to the one-dimensional Fourier or wavelet transforms. Both are integrals of paths. However, in reality these transforms are fundamentally different. Both the Fourier and wavelet transforms are linear transforms, and operate on each channel of the input path separately. In doing so they model the path as a linear combination of elements from some basis.
% %
% %Conversely, the signature transform is a nonlinear transform - indeed, it is a universal nonlinearity - and operates by combining information between different channels of the input path. In doing, the signature transform models \emph{functions of the path}; the universal nonlinearity property says that in some sense it provides a basis for such functions.

% %\subsubsection{Computing the signature transform}
% %Continuous piecewise linear paths are the paths of choice, computationally speaking, due to the fact that this is the only case for which efficient algorithms for computing the signature transform are known \cite{signatory}.
% %
% %This is not a serious hurdle when one wishes to compute the signature of a path $f$ that is not piecewise linear -- as the signature of piecewise linear approximations to $f$ will tend towards the signature of $f$ as the quality of the approximation increases -- but it does enforce this requirement on our imputation schemes.
% %
% %Thus all of the imputation schemes we examine will first seek to select a collection of points in data space (not necessarily only where we had data before), and then linear imputation will be performed to join them up into a piecewise linear path.

% %%%%%%%%%%%%%%%%%%%%%%%%%%%%%%%%%%%%%%%%%%%%%%%%%%%%%%%%%%%%%%%%%%%%%%%%
% \subsection{Notation}
% %%%%%%%%%%%%%%%%%%%%%%%%%%%%%%%%%%%%%%%%%%%%%%%%%%%%%%%%%%%%%%%%%%%%%%%%

% Given some set $A$, let the space of time series over $A$ be defined by
% \begin{equation}
%     \seriesspace(A) = \{((t_1, x_1), \ldots, (t_n, x_n)) \,\vert\, t_i \in \reals, x_i \in A, n \in \naturals, \text{ such that } t_1 \leq \cdots \leq t_n\}.\label{eq:seriesspace}
% \end{equation}
% %
% Note the subtle point that the $t_i$ are separated by $\leq$, not $<$.
% We make this slight change because it will later prove important for one
% of our studied imputation schemes. For example, contrast the
% corresponding definition in \citet[Section 1]{toth2019gp}.

% \paragraph{Time series and labels.}
% %
% Let $\lspace$ be a set and let $\dataspace_j = \reals$ for $j \in \{1,
% \ldots, d\}$ and $d \in \naturals$. Then we assume that we observe
% a dataset of labelled time series $(\mathbf{x}_k, y_k)$ for $k \in \{1,
% \ldots, N\}$, where $\mathbf{x}_k \in \seriesspace(\dataspace^*)$ and
% $y_k \in \lspace$, with $\dataspace^* = \prod_{j = 1}^d(\dataspace_j
% \cup \{*\})$ and $*$ representing no observation.
% %
% We similarly define $\dataspace = \prod_{j = 1}^d\dataspace_j$. Thus,
% $\dataspace$ is the data space, while $\dataspace^*$ is the data space
% allowing missing data, and $\lspace$ is the set of labels.

% % \subsubsection{Terminology for imputation schemes}
% % To avoid ambiguity, we will refer to standard imputation schemes, such as forward-fill, as a \emph{data-imputation} scheme, meaning to impute missing data points. In contrast we will use \emph{path-imputation} to describe the more general task of imputing the full continuous path.

% \subsection{Gaussian process adapter}\label{section:gpadapter}

% Some~(but not all) of the imputation schemes we consider are based on
% the uncertainty aware-framework of multi-task Gaussian process
% adapters~\citep{li2016scalable, futoma2017mgp}.
% %
% Let $\mathcal{W}, \mathcal{H}$ be some sets. Let $\ell \colon \lspace
% \times \lspace \to [0, \infty)$ be a loss function. Let $F \colon
% \dataspace^{[a, b]} \times \mathcal{W} \to \lspace$, be some (typically
% neural network) model, with $\mathcal{W}$ interpreted as a space of
% parameters. Let \begin{align*} \mu \colon [a, b] \times
% \seriesspace(\dataspace^*) \times \mathcal{H} &\to \dataspace\\ \Sigma
% \colon [a, b] \times [a, b] \times \seriesspace(\dataspace^*) \times
% \mathcal{H} &\to \dataspace    \end{align*} be mean and covariance
% functions, with $\mathcal{H}$ interpreted as a space of
% hyperparameters. The dependence on $\seriesspace(\dataspace^*)$ is to
% represent conditioning on observed values.

% Then the goal is to solve
% \begin{equation}\label{eq:gp-mc}
% \argmin_{\mathbf{w} \in \mathcal{W},\bm{\eta} \in \mathcal{H}} \sum_{k=1}^N \overbrace{\rule{0pt}{0.5cm} \mathbb{E}_{\mathbf{z}_k \sim \mathcal{N}\left( \mu(\cdot, \mathbf{x}_k, \eta), \Sigma(\cdot, \cdot, \mathbf{x}_k, \eta)\right) } \big[ \ell(F(\mathbf{z}_k, \mathbf{w}),
% y_k) \big] }^{\text{$E_k$}}.
% \end{equation}
% %
% As this expectation is typically not tractable, it is estimated by Monte
% Carlo (MC) sampling with $S$ samples, i.e.\
% \begin{equation}
% E_k \approx \frac{1}{S} \sum_{s=1}^{S} \ell(F(\mathbf{z}_{s, k}, \mathbf{w}), y_k),
% \end{equation}
% %
% where
% %
% \begin{equation}
%     \mathbf{z}_{s, k} \sim \mathcal{N}\left( \mu(\,\cdot\,, \mathbf{x}_k, \eta), \Sigma(\,\cdot\,, \,\cdot\,, \mathbf{x}_k, \eta)\right).
% \end{equation}

% Alternatively, one may forgo allowing the uncertainty to propagate through $F$ by instead passing the posterior mean directly to $F$; this corresponds to solving
% \begin{equation}\label{eq:gp-mean}
% \argmin_{\mathbf{w} \in \mathcal{W},\bm{\eta} \in \mathcal{H}} \sum_{k=1}^N \ell(F(\mu(\,\cdot\,,\mathbf{x}_k, \eta), \mathbf{w}), y_k).
% \end{equation}


\begin{comment}
\subsubsection{Application beyond signatures}

Whilst the task of converting observed data into a path in data space is particularly important for signatures, it also arises in the context of, for example, convolutional and recurrent neural networks.

Convolutions are often thought of in terms of discrete sums, but they are perhaps more naturally described as the integral cross-correlation between the underlying data path $f$ and the learnt filter $g_\theta$. Given sample points $t_1, \ldots, t_n \in [0, T]$, this integral is then approximated via numerical quadrature:
\begin{equation*}
    \frac{1}{T}\int_0^T f(t) g_\theta(t) \mathrm{d}t \approx \frac{1}{n}\sum_{i = 1}^n f(t_i) g_\theta(t_i),
\end{equation*}
although the $1/n$ scaling is really only justified in the case that the $t_i$ are equally spaced.\footnote{The $g_\theta$ is typically a step function in `normal' convolutional layers. Some works exists on replacing it with e.g. B-splines \cite{fey2018splinecnn} to better handle irregular data. The oddity of scaling by $1/n$ with irregular data has not been explicitly addressed in the literature, at least to our knowledge; indeed quite conversely we have seen it used without remark.} Thus we see that with convolutions, we are implicitly interpreting the observed data as a path in data space.

Similarly, the connection between dynamical systems and recurrent neural networks are well known \cite{FUNAHASHI1993801, continuousrnn}, and these tend to use a similar setup.

For non-signature methods as for signature methods, this implicit usage of data as a path in data space often seems to be swept under the rug, and we suspect it is one deserving further attention.
\end{comment}



% %%%%%%%%%%%%%%%%%%%%%%%%%%%%%%%%%%%%%%%%%%%%%%%%%%%%%%%%%%%%%%%%%%%%%%%%
% \section{Related work} %Artemiss
% %%%%%%%%%%%%%%%%%%%%%%%%%%%%%%%%%%%%%%%%%%%%%%%%%%%%%%%%%%%%%%%%%%%%%%%%

% A key motivation for this work is the use of the signature transform in
% machine learning: recent work~\citep{primer2016,
% kormilitzlin2016, yang2016rotation, li2017lpsnet, yang2017leveraging,
% PerezArribas2018, morrill2019sepsis} typically employed the signature transform
% as a nonparametric feature extractor, on top of which
% a model is learnt. Integrations into typical neural networks are also being
% actively discussed~\citep{jeremythesis, logsigrnn, kidger2019deep}, as
% well as kernel-based approaches~\citep{chevyrev2018signature, kiraly2019kernels}.
% \citep{toth2019gp} show how this kernel may be used to define a Gaussian process. To our
% knowledge, no prior work has regarded path calculation as an \emph{imputation
% problem}; typically, data is converted into a continuous path via
% linear/rectilinear interpolation~\citep{primer2016,
% fermanian2019embedding}, or a hybrid of the two~\citep{levin2013}.
% %
% The general problem of imputing data is well-known and well-studied~\citep[Chapter 25]{gelman2007dataanalysis}.
% %
% % Common simple imputation schemes are to impute missing values as the
% % mean of its channel, or to forward fill the last observed value,
% % although it has been observed that incautious use of such methods can
% % result in heavily biased models \cite{Molnar2008}.
% %
% Imputation methods typically only fill in missing discrete data
% points, and do not attempt to impute the underlying continuous path.
% Gaussian process adapters~\citep{li2016scalable}, by contrast, are
% capable of imputing a \emph{full} continuous path, from which we
% may sample arbitrarily. Hence, this framework also needs to be
% considered in this paper.



\section{Path imputations for signature models}

Signatures act on continuous paths. However, in real-world
applications, temporal data typically appears as a discretised
collection of measurements, potentially irregularly-spaced and
asynchronously observed. To apply the signature to this data,
it first has to be imputed into a continuous path.
%
We believe this step to have a significant impact on the
resulting signature, and thus also on models employing the signature.
To assess this hypothesis, we explicitly treat this transformation
as a \emph{path imputation}, i.e.\ a mapping of the form $ \phi \colon \seriesspace(\dataspace^*) \to (\reals \times \dataspace)^{[a, b]}.$
%
We aim to learn a function $ g \colon \seriesspace(\dataspace^*) \to
\lspace$, which decomposes to $g = F \circ \phi$, where $F$ refers to
a classifier, mapping from $(\reals \times \dataspace)^{[a, b]} \times \mathcal{W} $ to
$\lspace$. Given a loss function $\ell$ and a set of $p$ path
imputation strategies, ${\Phi} =(\phi_i)_{i=1}^p$, we seek to minimise the
objective:
%
\begin{equation}
    \argmin_{\phi_i \in \Phi, \mathbf{w} \in \mathcal{W}} \quad \mathbb{E}_{(\mathbf{x},y) \sim P(\seriesspace(\dataspace^*),\lspace) } \left[ \ell( g(\mathbf{x}; \phi_i, \mathbf{w}), y) \right]
    \label{eq:objective}
\end{equation} 
%
Even though Equation \ref{eq:objective} could be formulated more
\emph{implicitly}~(i.e.\ without any explicit imputation step),
this formulation enables us to make explicit how the signature transform
`interprets' the raw data for downstream classification tasks. 

\paragraph{Path imputation strategies}
We consider the following set of strategies for path imputation, i.e.\
\begin{inparaenum}[(1)]
    \item linear interpolation (lin),
    \item forward filling (ff),
    \item indicator imputation (ind),
    \item zero imputation (zero),
    \item causal imputation\footnote{This strategy is similar to the time-joined transformation \citep{levin2013}. For more details, please refer to Section \ref{sec:Causal signature imputation} in the appendix.} (causal), and
    \item Gaussian process adapters (GP).
\end{inparaenum}
%
Strategies 1--5 can be seen as a fixed preprocessing step, whereas GP
adapters~(strategy~6) are optimised end-to-end with the downstream task. For more
details regarding these strategies, please refer to Section \ref{supp: Imputation} in the
appendix.
\paragraph{GP adapter with posterior moments}
\begin{figure}[t]
    \centering
    \hspace*{1cm}
    \includegraphics[width=0.7\columnwidth]{figures/overview.pdf}
    \caption{Overview of our proposed extension of GP adapters, GP-PoM, leveraging both posterior moments (mean and variance). In comparison, the conventional GP adapter feeds MC samples (faded colors in the background) drawn from the GP posterior into the classifier}
    \label{fig:overview}
\end{figure}

For conventional GP adapters, one major drawback with the formulations
of \citet{li2016scalable} and \citet{futoma2017mgp}, as described in
equation \eqref{eq:gp-mc}, is that approximating the expectation outside
of the loss function with MC sampling is expensive.
%
During prediction, \citet{li2016scalable} proposed to overcome this
issue by sacrificing the uncertainty in the loss function and to simply
pass the posterior mean, as in supplementary Equation
\eqref{eq:gp-mean}\footnote{%
  Equations \eqref{eq:gp-mc} and \eqref{eq:gp-mean} are of course not in
  general equal, so following \citet{futoma2017mgp}, our standard GP
  adapter uses MC sampling both in training and testing.
}.
%

To address both points, we propose to instead also pass the posterior
covariance of the Gaussian process to the classifier $F$. This saves the
cost of MC sampling whilst explicitly providing~$F$ with uncertainty
information during the prediction\footnote{Even if MC sampling
is used during prediction, $F$ has no per-sample access to
uncertainty about the imputation.}.
%
However, the full covariance matrix may become very large, and it is
not obvious that all interactions are relevant to the subsequent
classifier. \newline 
This is  why we simplify matters by taking the posterior
\emph{variance} at every point, and concatenate it with the posterior
mean at every point, to produce a path whose evolution also captures the
uncertainty at every point:
%
%\vspace{-0.08cm}
\begin{align}
    \tau &\colon [a, b] \times \seriesspace(\dataspace^*) \times \mathcal{H} \to \dataspace \times \dataspace\\
    \tau &\colon t, \mathbf{x}, \eta \mapsto (\mu(t, \mathbf{x}, \eta), \Sigma(t, t, \mathbf{x}, \eta)).
\end{align}
%
This corresponds to solving
\begin{equation}\label{eq:gp-moments}
\argmin_{\mathbf{w} \in \mathcal{W},\bm{\eta} \in \mathcal{H}} \sum_{k=1}^N \ell(F(\tau(\,\cdot\,,\mathbf{x}_k, \eta), \mathbf{w}), y_k),
\end{equation}
%
where instead now $F \colon (\dataspace \times \dataspace)^{[a, b]}
\times \mathcal{W} \to \lspace$.
%
%
Figure~\ref{fig:overview} depicts our proposed strategy \textsc{GP-PoM}. 

In our context of interest, when $F$ is a signature model, it is now straightforward to
compute the signature of the Gaussian process, simply by querying many
points to construct a piecewise linear approximation to the process.
The choice of kernel has nontrivial mathematical implications for this procedure: for example if a Mat{\'e}rn 1/2 kernel is chosen, then the resulting path is not of bounded variation and the definition of the signature transform given in equation \eqref{eq:signature} does not hold, and rough path theory \cite{lyons1998differential} must instead be invoked to define the signature transform.

% %%%%%%%%%%%%%%%%%%%%%%%%%%%%%%%%%%%%%%%%%%%%%%%%%%%%%%%%%%%%%%%%%%%%%%%%%%%%%%%%%%%%%%%%%%%%%%%%%%%
% \section{Path imputations} %OLD
% %%%%%%%%%%%%%%%%%%%%%%%%%%%%%%%%%%%%%%%%%%%%%%%%%%%%%%%%%%%%%%%%%%%%%%%%%%%%%%%%%%%%%%%%%%%%%%%%%%%

% %For simplicity of presentation, we will now assume that $\dataspace_j = \reals$.%, although in principle the theory actually only requires that $\dataspace_j$ is a Banach space.

% Our goal is to learn a function from $\seriesspace(\dataspace^*)$ to
% $\lspace$. Motivated by signature theory, we believe that this is best
% done by selecting a path imputation strategy
% % Our belief (perhaps merely motivated by signature theory, but in this paper explicitly because we wish to apply signatures), is that this is sensibly done by selecting a path-imputation strategy
% %
% \begin{equation}
%     \phi \colon \seriesspace(\dataspace^*) \to (\reals \times \dataspace)^{[a, b]},
% \end{equation}
% and only \emph{afterwards} learning a map from $\dataspace^{[a, b]}$ to $\lspace$.
% %
% Explicitly, we seek $\phi$ such that
% \begin{equation}\label{eq:phi}
% \phi(\mathbf{x}) = f,    
% \end{equation}
% % TODO: can we compress these equations somewhat?
% where
% $\mathbf{x} = ((t_1, x_1), \ldots, (t_n, x_n)) \in
% \seriesspace(\dataspace^*)$
% and each $x_i = (x_i^1, \ldots, x_i^d) \in \dataspace^*$, % = \prod_{j = 1}^d (\dataspace_j \cup \{*\}),
% with $f \colon [a, b] \to \reals \times \dataspace$ being a continuous,
% piecewise linear function such that there exist points
% \begin{equation}\label{eq:ss}
% a = s_1 < s_2 < \cdots < s_n = b    
% \end{equation}
% so that $f(s_i) = (t_i, z_i)$, with $z_i^j = x_i^j$ for all $x_i^j \neq
% *$~(if $x_i^j = *$, we impose no constraints).
% %
% Furthermore, we impose that $f$ must be monotonically nondecreasing in its $t$ output.

% % Two remarks on this problem formulation.

% % First, we emphasise that the time series does not necessarily have to irregularly sampled or partially observed, although this is the most general setting. Even with a regularly spaced and fully observed time series, we must still path-impute the rest of the continuous path at the values in between.

% Note that we have explicitly \emph{not} formulated this as seeking
% a function $f \colon [t_1, t_n] \to \reals \times \dataspace$ with
% $f(t_i) = (t_i, z_i)$ and $z_i^j = x_i^j$ for $x_i^j \neq *$, as might
% na{\"i}vely be assumed. This alternate formulation is essentially what
% is typically used; see for example the `time embedding' of
% \citet{fermanian2019embedding}. However, the slight extra generality in
% our formulation will be precisely what is necessary to solve the flaws
% in existing schemes that we are about to identify.

% \subsection{Flaws in existing imputation schemes}\label{section:flaws}

% We move on to discussing the flaws of existing schemes, namely
% %
% \begin{inparaenum}[(i)]
%   \item their fragile dependence on sampling in unrelated channels, and
%   \item their non-causality,
% \end{inparaenum}
% %
% before subsequently addressing them.

% %%%%%%%%%%%%%%%%%%%%%%%%%%%%%%%%%%%%%%%%%%%%%%%%%%%%%%%%%%%%%%%%%%%%%%%%
% \subsubsection{Fragile dependence on sampling in unrelated channels}
% %%%%%%%%%%%%%%%%%%%%%%%%%%%%%%%%%%%%%%%%%%%%%%%%%%%%%%%%%%%%%%%%%%%%%%%%

% Na{\"i}vely applying standard data-imputation schemes will typically
% result in the imputed path exhibiting an undesired dependence on the
% observations. This can result in dramatic changes in the
% areas-against-time computed by the signature~(potentially going so far
% as to even change the sign of the result).
% %
% % TODO: update figure to include time series?
% Figure~\ref{fig:bentline} describes this problem schematically; please
% refer to Supplementary Section~\ref{sec:Fragile dependence} for more
% details.
% %
% % TODO: can we make this into a theorem?
% In general, we can summarise this by saying that \emph{when used with the signature transform, any traditional data-imputation scheme will have a fragile dependence on unrelated observations}.

% \begin{figure}[t]
%     \centering
%     \includegraphics[width=0.45 \columnwidth]{figures/sig_path3.pdf}
%     %\vspace{-1.5em}
%     \caption{Levy area of forward-fill imputed path.}\label{fig:bentline}
%     \vspace{-1em}
% \end{figure}

% %%%%%%%%%%%%%%%%%%%%%%%%%%%%%%%%%%%%%%%%%%%%%%%%%%%%%%%%%%%%%%%%%%%%%%%%
% \subsubsection{Non-causality}
% %%%%%%%%%%%%%%%%%%%%%%%%%%%%%%%%%%%%%%%%%%%%%%%%%%%%%%%%%%%%%%%%%%%%%%%%

% Once fully observed data has been acquired, either by observation or
% data-imputation, then our next operation is always to perform linear
% path-imputation.
% %
% However, linear path-imputation is explicitly non-causal.
% %
% Given some $(t_i, x_i)$ and $(t_{i+1}, x_{i + 1})$, then the
% interpolating function $f$, considered on the interval $(t_i, t_{i
% + 1})$, will already be moving toward $x_{i + 1}$ before that piece of
% data has arrived:
% %
% \begin{equation}
%     f(t) = x_{i} \frac{t_{i + 1} - t}{t_{i + 1} - t_i} + x_{i + 1} \frac{t - t_i}{t_{i + 1} - t_i}\text{ for }t \in (t_i, t_{i + 1})
% \end{equation}
% %
% This poses an issue in, for example, the online setting.
% %
% Non-causality is not always a concern, but we will see in the
% next section that the same strategy for overcoming the previous flaw can
% overcome this challenge too.
% %
% % TODO: can we make this into a theorem?
% We can summarise this by saying that \emph{when used with the signature transform, any traditional data-imputation scheme will be non-causal}.
% % TODO: try and adjust this, it's really the path imputation that's noncausal.

% %%%%%%%%%%%%%%%%%%%%%%%%%%%%%%%%%%%%%%%%%%%%%%%%%%%%%%%%%%%%%%%%%%%%%%%%
% \subsection{Causal signature imputation}
% %%%%%%%%%%%%%%%%%%%%%%%%%%%%%%%%%%%%%%%%%%%%%%%%%%%%%%%%%%%%%%%%%%%%%%%%

% We have spoken so far about the limitations of traditional
% data imputation schemes, and at first glance, one may be forgiven for
% thinking that these are issues that are unavoidable. 
% %
% However, it turns out that we need not be limited just to these
% traditional imputation schemes. The trick is to consider time not as
% a \emph{parameterization}, but as a \emph{channel}. Contrast the two
% possible formulations that were discussed for $\phi$ defined in equation
% \eqref{eq:phi}.\footnote{To be clear, using time as a channel is already
% a well-known trick in the signature literature that we do not take
% credit for inventing! See for example \citet[Definition
% A.3]{kidger2019deep}. It is however pleasing that something commonly
% used in the theory of signatures is also what allows us to overcome what
% we identify as some of their limitations.}
% %
% This leads to our novel `meta imputation strategy', which we refer to as
% \emph{causal signature imputation}. It will turn any traditional causal
% data-imputation strategy (for example, feed-forward) into a causal
% path-imputation strategy for signatures; at the same time it will
% overcome the issue of a fragile dependence.

% Suppose we have $\mathbf{x} \in \seriesspace(\dataspace^*)$, and some
% favourite choice of causal data imputation strategy $c \colon
% \seriesspace(\dataspace^*) \to \seriesspace(\dataspace)$.
% %
% Next, given $\mathbf{x} = ((t_1, x_1), \ldots, (t_n, x_n)) \in \seriesspace(\dataspace)$,
% %
% we define the operation $\Phi\colon \seriesspace(\dataspace) \to
% \seriesspace(\dataspace)$ by
% %
% \begin{align}
%     \Phi(\mathbf{x}) = (&(t_1, x_1), (t_2, x_1), (t_2, x_2),(t_3, x_2),\nonumber\\
%     &\ldots,\nonumber\\
%     &(t_i, x_i), (t_{i + 1}, x_i), (t_{i + 1}, x_{i + 1}), (t_{i + 2}, x_{i + 1}),\nonumber\\
%     &\ldots,\nonumber\\
%     &(t_{n - 1}, x_{n - 1}), (t_n, x_{n - 1}),(t_n, x_n)).\label{eq:causalsig}
% \end{align}
% %
% That is, \emph{first} time is updated, and \emph{then} the corresponding
% observation in data space is updated. This means that the change in data
% space occurs instantaneously.
% %
% For each $n \in \naturals$ (and given $a < b$), fix any $s_i^{(n)}$ for $i \in \{1, \ldots, n \}$ as in equation \eqref{eq:ss}. (We will see that the exact choice is unimportant in a moment.) Given
% \begin{align*}
%     \mathbf{x} = ((t_1, x_1), \ldots, (t_n, x_n)) \in \seriesspace(\dataspace),
% \end{align*}
% let $\psi \colon \seriesspace(\dataspace) \to (\reals \times \dataspace)^{[a, b]}$ be the unique continuous piecewise linear path such that $\psi(s_i^{(n)}) = (t_i, x_i)$. Note that this is just a slight generalisation of the linear path-imputation that has already been performed so far; we are simply no longer asking for additional assumptions of the form $s_i^{(n)} = t_i$.\footnote{As in the $\mathbf{\varphi}_\theta$ of \cite{toth2019gp}, for example.}
% %
% Finally, we put this all together, and define the causal signature imputation strategy $\phi_c$ associated with $c$ to be
% %
% \begin{equation}
% \phi_c = \psi \circ \Phi \circ c,
% \end{equation}
% %
% which will be a map $\seriesspace(\dataspace^*) \to (\reals \times \dataspace)^{[a, b]}$.
% %
% Thus $\phi_c$ defines a family of path-imputation schemes, parameterized
% by a choice of data-imputation scheme. Please refer to Supplemental
% Section~\ref{sec:Causal signature imputation} for more details and
% a worked-out example.

% \paragraph{Comparison to known signature imputation strategies.}
% %
% Our formulation can be related to two operations in the signature
% literature. First is the \emph{lead-lag} transform \cite{primer2016}.
% With the lead-lag transform, the entire path is \emph{duplicated}, after
% which each side is alternately updated.
% %
% Conversely, in causal signature imputation, the path is instead
% \emph{split} between $t$ and $(x^1, \ldots, x^n)$, and then each side is
% alternately updated.
% %
% Second is the comparison to the linear and rectilinear embedding
% strategies, see for example \citet{fermanian2019embedding}. It is
% possible to interpret $\psi \circ \Phi$ as a hybrid between the linear
% and rectilinear embeddings: it is rectilinear with respect to an
% ordering of $t$ and $(x^1, \ldots, x^n)$, and linear on $(x^1, \ldots,
% x^n)$.

% \subsection{Extending Gaussian process adapters with posterior moments}\label{section:ourgpadapter}
% TODO: % emphasize that the idea behind the PoM is that the model has access to uncertainty information during the prediction step. We argue that the previous formulation is not really uncertainty-aware in this sense, and we propose an approach to fix this. Notably, PoM comes at the cost of losing uncertainty \emph{about} the models predictions. 

% Non-causality is not necessarily a worrying flaw, especially when not operating in an online setting. As such, we go on to consider another promising, but non-causal, path-imputation method, by adapting the Gaussian process adapter described in section \ref{section:gpadapter}.

% One major drawback with the formulations of \cite{li2016scalable} and \cite{futoma2017mgp}, as described in equation \eqref{eq:gp-mc}, is that approximating the expectation outside of the loss function with Monte Carlo sampling is expensive.

% During prediction, they propose to overcome this issue by sacrificing the uncertainty in the loss function and to simply pass the posterior mean, as in \eqref{eq:gp-mean}.\footnote{Equations \eqref{eq:gp-mc} and \eqref{eq:gp-mean} are of course not in general equal, so really the same procedure should be used for both training and test -- despite the description given in \cite[Section 3.1]{li2016scalable}.}

% To address both points, we propose to instead also pass the posterior covariance of the Gaussian process to the classifier $F$. This saves the cost of Monte Carlo sampling whilst explicitly providing the classifier with $F$ with uncertainty information.

% However, full covariance matrices may become very large, with much of it not obviously helpful to the subsequent classifier. We instead simplify matters by taking the posterior variance at every point, and concatenate it with the posterior mean at every point, to produce a path whose evolution describes the uncertainty at every point:
% \begin{align*}
%     \tau &\colon [a, b] \times \seriesspace(\dataspace^*) \times \mathcal{H} \to \dataspace \times \dataspace\\
%     \tau &\colon t, \mathbf{x}, \eta \mapsto (\mu(t, \mathbf{x}, \eta), \Sigma(t, t, \mathbf{x}, \eta)).
% \end{align*}

% This corresponds to solving
% \begin{equation}\label{eq:gp-moments}
% \argmin_{\mathbf{w} \in \mathcal{W},\bm{\eta} \in \mathcal{H}} \sum_{k=1}^N \ell(F(\tau(\,\cdot\,,\mathbf{x}_k, \eta), \mathbf{w}), y_k),
% \end{equation}
% where instead now
% \begin{equation*}
%     F \colon (\dataspace \times \dataspace)^{[a, b]} \times \mathcal{W} \to \lspace.
% \end{equation*}

% Training this model is no harder than the formulation of \cite{li2016scalable}. Abusing notation by conflating the Gaussian process sample $\mathbf{z}$ with its finite dimensional distribution that is of interest, simply decompose $\mathbf{z} = \mu + R \xi$, with $\xi$ a standard multivariate normal and $R$ such that $\Sigma = RR^T$. The derivatives of equation \eqref{eq:gp-moments} are now no harder to describe than they were before.

% In our context of interest, when $F$ is a signature model, then it is now straightforward to compute the signature of the Gaussian process, simply by sampling many points to construct a piecewise linear approximation to the process. Technically speaking, the choice of kernel has nontrivial mathematical implications for this procedure: for example if a Mat{\'e}rn 1/2 kernel is chosen, then the resulting path is not of bounded variation and the definition of the signature transform given in equation \eqref{eq:signature} does not hold, and rough path theory \cite{lyons1998differential} must instead be invoked to define the signature transform.

% This formulation, whilst lacking causality, is nonetheless robust to the issue of fragility with respect to observations in unrelated channels. The Gaussian process adapter operates globally on all observed data, and observations in unrelated channels are allowed to be uncorrelated. This framework is also naturally more powerful at modelling than the explicit parameter-free schemes discussed thus far.


%%%%%%%%%%%%%%%%%%%%%%%%%%%%%%%%%%%%%%%%%%%%%%%%%%%%%%%%%%%%%%%%%%%%%%%%
\section{Experiments}
%%%%%%%%%%%%%%%%%%%%%%%%%%%%%%%%%%%%%%%%%%%%%%%%%%%%%%%%%%%%%%%%%%%%%%%%

We first introduce our experimental setup~(datasets and model
architectures) before presenting and discussing quantitative results.

%%%%%%%%%%%%%%%%%%%%%%%%%%%%%%%%%%%%%%%%%%%%%%%%%%%%%%%%%%%%%%%%%%%%%%%%
\paragraph{Datasets and preprocessing}
%%%%%%%%%%%%%%%%%%%%%%%%%%%%%%%%%%%%%%%%%%%%%%%%%%%%%%%%%%%%%%%%%%%%%%%%

We analyse four real-world time series datasets, i.e.\
%
\begin{inparaenum}[(i)]
  \item \texttt{Physionet2012}~\citep{goldberger2000physiobank},
  \item \texttt{PenDigits}~\citep{Dua2019},
  \item \texttt{LSST}~\citep{allam2018photometric}, and
  \item \texttt{CharacterTrajectories}~\citep{Dua2019}.
\end{inparaenum}
%
For dataset statistics and necessary filtering steps, please refer to Section \ref{supp: Dataset stats} in the appendix.
Moreover, to efficiently compute the signature, we
sample the imputed path in a \emph{fixed} time resolution\footnote{For Physionet2012 hourly, for the other datasets once per originally ovserved time step}, resulting in
a piecewise linear path . \newline
For time series that are not irregularly spaced, we apply two types of random subsampling as an additional
preprocessing step for all but the \texttt{Physionet2012} dataset,
namely
\begin{inparaenum}[(1)]
    \item `Random': Missing at random; on the instance level, we discard 50\% of all observations.
    \item `Label-based': Missing not at random; for each
      class, we uniformly sample missingness frequencies between $40\%$
      and $60\%$.
\end{inparaenum}
%
Since \texttt{PenDigits} consists of particularly short time series~(8
steps, 2 dimensions), we use more moderate frequencies of $30\%$
and $20$--$40\%$, respectively, for discarding observations.

\paragraph{Models}
%
We study the following models:
\begin{inparaenum}[(1)]
  \item \textsc{Sig}, a simple signature model that involves a linear
    augmentation, the signature transform~(signature block) and a final
    module of dense layers,
  %
  \item \textsc{RNN}, an RNN model using GRU cells~\citep{cho2014learning},
  %
  \item \textsc{RNNSig}, which extends the signature transform to a window-based
    stream of signatures, and where the final neural module is a GRU
    sliding over the stream of signatures, and
  %
  \item \textsc{DeepSig}, a deep signature model sequentially employing two signature blocks featuring augmentation and signature transforms,
  following \citet{kidger2019deep}.
\end{inparaenum}
%
\newline
Please refer to Supplementary Section~\ref{supp: Model Architectures} for more details about the
architectures and implementations. We use the `Signatory' package to
calculate the signature transform~\citep{signatory}, and implemented all GP
adapters in the `GPyTorch' framework~\citep{gardner2018gpytorch}.

\paragraph{Training and evaluation}
%
We use the predefined training and testing splits for each dataset,
separating $20\%$ of the training split as a validation set for
hyperparameter tuning.
%
For each setting, we run a randomised hyperparameter search of $20$
calls and train each of these fits until convergence~(at most 100
epochs; we stop early if the performance on the validation
split does not improve for 20 epochs). As for performance metrics, for
binary classification tasks, we optimize average precision and also report AUROC. For multi-class
classification, we optimize balanced accuracy~(BAC) and additionally report accuracy and weighted AUROC (w-AUROC)\footnote{AUROC is computed for each label and averaged as weighted by the support of each class}.
%
Having selected the best hyperparameter configuration for each setting,
we repeat $5$ fits, per fit select the best model state in terms of the best validation performance, and finally report mean and
standard deviation (error bars) of the performance metrics on the testing split.

\paragraph{Results}
%
\begin{figure*}[tbp]
  \subcaptionbox{CharacterTrajectories-R}{%
    \includegraphics[width=0.48\linewidth]{plots/barplot_main_CharacterTrajectories-MissingAtRandomSubsampler.pdf}\quad%
%     \includegraphics[width=0.48\linewidth]{plots/barplot_main_params_CharacterTrajectories-MissingAtRandomSubsampler.pdf}%
   }%
  \subcaptionbox{CharacterTrajectories-L}{%
    \includegraphics[width=0.48\linewidth]{plots/barplot_main_CharacterTrajectories-LabelBasedSubsampler.pdf}\quad%
    % \includegraphics[width=0.48\linewidth]{plots/barplot_main_params_CharacterTrajectories-LabelBasedSubsampler.pdf}%
  }
  \caption{Experimental Results visualized for Character Trajectories Dataset.
  %The rows indicate the subsampling type: Random (R) versus Label-based (L).
  The bars display performance in terms of balanced accuracy (BAC), whereas the panels indicate the used subsampling. Left: Random (R), right: label-based (L).}
  \label{fig:results_main}
\end{figure*}

In Tables \ref{tab:character} and \ref{tab:pendigits}, the results for \texttt{Character Trajectories} and \texttt{PenDigits} under label-based subsampling are shown, respectively. For the remaining datasets and subsamplings, please refer to the Tables \ref{supp: tab character}--\ref{supp: tab physionet} in the appendix. 
% In addition, \ref{fig:results_main} depicts a barplot visualization of the results for the \texttt{Character Trajectories} dataset~(please refer to supplemental Figure~\ref{fig:supp_results} for the other datasets).
We observe that both \textsc{DeepSig} as well as the signature-free \textsc{RNN} perform well
over many scenarios. In particular, they are robust to various
imputation schemes. However, we also see that certain signature models,
in particular \textsc{Sig}, are heavily impacted by the choice of
imputation strategy. Figure \ref{fig:results_main} exemplifies this finding in a barplot visualization; for the remaining visualisations including the number of parameters of the optimized models, please refer to the Supplementary Figures \ref{supp: barplots1} and \ref{supp: barplots2}. In the case of \texttt{Character Trajectories},
\textsc{Sig} was only able to achieve acceptable performance through
our novel \textsc{GP-PoM} strategy. In \texttt{PenDigits}, we encountered
issues of numerical stability for the original GP adapter\footnote{They
were addressed by jittering the diagonal in the Cholesky
decomposition.}; not so for \textsc{GP-PoM}. Furthermore, we found that \textsc{GP-PoM} tends to converge faster to a better performance than the original GP adapter, as exemplified in Supplementary Figure \ref{supp: gp-training}.

\begin{table}[h]
    \caption{\textbf{CharacterTrajectories} Dataset under label-based subsampling. The top three methods are highlighted: bold \& underlined, bold, underlined.}
    \centering
    %\small{
    \begin{tabular}{llll}
\toprule
\nicefrac{\text{model}}{\text{metric}} &                                        w-AUROC &                                            BAC &                                       Accuracy \\
%model          &                                                &                                                &                                                \\
          &                                                &                                                &                                                \\
\midrule
GP-PoM-DeepSig &                           $ 99.582 \pm 0.671 $ &                           $ 95.155 \pm 1.501 $ &                           $ 94.958 \pm 1.716 $ \\
GP-PoM-RNN     &            $  \underline{ 99.973 \pm 0.015 } $ &               $  \mathbf{ 98.161 \pm 0.664 } $ &               $  \mathbf{ 98.273 \pm 0.602 } $ \\
GP-PoM-RNNSig  &                           $ 99.696 \pm 0.089 $ &                           $ 92.778 \pm 1.239 $ &                           $ 93.231 \pm 1.133 $ \\
GP-PoM-Sig     &                           $ 99.516 \pm 0.075 $ &                           $ 88.627 \pm 1.416 $ &                           $ 89.011 \pm 1.319 $ \\
GP-DeepSig     &                           $ 99.290 \pm 0.704 $ &                           $ 89.545 \pm 2.996 $ &                           $ 89.368 \pm 3.123 $ \\
GP-RNN         &                           $ 99.970 \pm 0.011 $ &                           $ 97.712 \pm 0.266 $ &                           $ 97.873 \pm 0.251 $ \\
GP-RNNSig      &                           $ 96.669 \pm 2.393 $ &                          $ 65.717 \pm 13.691 $ &                          $ 67.052 \pm 13.182 $ \\
GP-Sig         &                           $ 95.283 \pm 1.602 $ &                           $ 62.423 \pm 6.110 $ &                           $ 63.614 \pm 5.958 $ \\
causal-DeepSig &                           $ 99.940 \pm 0.024 $ &                           $ 97.272 \pm 0.709 $ &                           $ 97.437 \pm 0.620 $ \\
causal-RNN     &                           $ 99.960 \pm 0.010 $ &                           $ 97.239 \pm 0.516 $ &                           $ 97.409 \pm 0.481 $ \\
causal-RNNSig  &                           $ 99.523 \pm 0.155 $ &                           $ 89.922 \pm 2.301 $ &                           $ 90.585 \pm 2.186 $ \\
causal-Sig     &                           $ 95.747 \pm 4.957 $ &                          $ 66.307 \pm 21.794 $ &                          $ 68.259 \pm 20.757 $ \\
ff-DeepSig     &                           $ 99.953 \pm 0.041 $ &                           $ 97.956 \pm 0.677 $ &                           $ 98.078 \pm 0.656 $ \\
ff-RNN         &                           $ 99.942 \pm 0.011 $ &                           $ 96.942 \pm 0.486 $ &                           $ 97.159 \pm 0.444 $ \\
ff-RNNSig      &                           $ 99.720 \pm 0.071 $ &                           $ 92.568 \pm 1.091 $ &                           $ 93.148 \pm 1.011 $ \\
ff-Sig         &                           $ 94.828 \pm 8.117 $ &                          $ 67.169 \pm 26.338 $ &                          $ 68.649 \pm 26.125 $ \\
ind-DeepSig    &  $  \mathbf{ \underline{ 99.988 \pm 0.013 }} $ &  $  \mathbf{ \underline{ 98.591 \pm 0.294 }} $ &  $  \mathbf{ \underline{ 98.719 \pm 0.263 }} $ \\
ind-RNN        &                           $ 99.916 \pm 0.020 $ &                           $ 96.414 \pm 0.406 $ &                           $ 96.671 \pm 0.367 $ \\
ind-RNNSig     &                           $ 99.802 \pm 0.032 $ &                           $ 93.787 \pm 0.463 $ &                           $ 94.234 \pm 0.442 $ \\
ind-Sig        &                          $ 91.661 \pm 10.003 $ &                          $ 56.423 \pm 22.796 $ &                          $ 58.384 \pm 22.932 $ \\
lin-DeepSig    &                           $ 99.970 \pm 0.010 $ &            $  \underline{ 98.051 \pm 0.743 } $ &            $  \underline{ 98.217 \pm 0.671 } $ \\
lin-RNN        &                           $ 99.880 \pm 0.059 $ &                           $ 96.906 \pm 1.314 $ &                           $ 97.117 \pm 1.196 $ \\
lin-RNNSig     &                           $ 99.876 \pm 0.035 $ &                           $ 94.848 \pm 0.916 $ &                           $ 95.292 \pm 0.842 $ \\
lin-Sig        &                          $ 80.442 \pm 18.228 $ &                          $ 31.193 \pm 23.962 $ &                          $ 32.326 \pm 24.679 $ \\
zero-DeepSig   &               $  \mathbf{ 99.977 \pm 0.010 } $ &                           $ 98.030 \pm 0.357 $ &                           $ 98.189 \pm 0.358 $ \\
zero-RNN       &                           $ 99.967 \pm 0.014 $ &                           $ 97.428 \pm 0.572 $ &                           $ 97.549 \pm 0.596 $ \\
zero-RNNSig    &                           $ 99.699 \pm 0.132 $ &                           $ 91.752 \pm 1.782 $ &                           $ 92.368 \pm 1.662 $ \\
zero-Sig       &                          $ 77.727 \pm 23.671 $ &                          $ 37.992 \pm 34.456 $ &                          $ 38.955 \pm 35.232 $ \\
\bottomrule
\end{tabular}

    %}
    \label{tab:character}
\end{table}
\begin{table}[h]
    \caption{\textbf{PenDigits} Dataset under label-based subsampling. The top three methods are highlighted: bold \& underlined, bold, underlined.}
    \centering
    %\small{
    \begin{tabular}{llll}
\toprule
metric &                                          w-AUROC &                                             BAC &                                        Accuracy \\
model          &                                                  &                                                 &                                                 \\
\midrule
GP-DeepSig     &                             $ 0.9289 \pm 0.015 $ &                            $ 0.6059 \pm 0.041 $ &                            $ 0.6048 \pm 0.041 $ \\
GP-PoM-DeepSig &            $  \underline{ 0.9993 \pm 0.00032 } $ &                 $  \mathbf{ 0.974 \pm 0.003 } $ &                $  \mathbf{ 0.9738 \pm 0.003 } $ \\
GP-PoM-RNN     &                            $ 0.999 \pm 0.00016 $ &                            $ 0.9635 \pm 0.003 $ &                            $ 0.9631 \pm 0.003 $ \\
GP-PoM-RNNSig  &                           $ 0.9967 \pm 0.00073 $ &                           $ 0.9302 \pm 0.0076 $ &                           $ 0.9297 \pm 0.0076 $ \\
GP-PoM-Sig     &                            $ 0.9915 \pm 0.0014 $ &                            $ 0.8809 \pm 0.015 $ &                              $ 0.88 \pm 0.015 $ \\
GP-RNN         &                             $ 0.9517 \pm 0.014 $ &                            $ 0.6754 \pm 0.048 $ &                            $ 0.6743 \pm 0.048 $ \\
GP-RNNSig      &                              $ 0.845 \pm 0.013 $ &                             $ 0.4218 \pm 0.02 $ &                            $ 0.4214 \pm 0.019 $ \\
GP-Sig         &                             $ 0.8031 \pm 0.027 $ &                            $ 0.3777 \pm 0.036 $ &                            $ 0.3772 \pm 0.036 $ \\
causal-DeepSig &                           $ 0.9924 \pm 0.00075 $ &                           $ 0.8962 \pm 0.0075 $ &                           $ 0.8951 \pm 0.0075 $ \\
causal-RNN     &                           $ 0.9924 \pm 0.00098 $ &                            $ 0.895 \pm 0.0048 $ &                            $ 0.8942 \pm 0.005 $ \\
causal-RNNSig  &                            $ 0.993 \pm 0.00041 $ &                           $ 0.8919 \pm 0.0048 $ &                           $ 0.8914 \pm 0.0049 $ \\
causal-Sig     &                           $ 0.9837 \pm 0.00065 $ &                             $ 0.832 \pm 0.004 $ &                           $ 0.8308 \pm 0.0043 $ \\
ff-DeepSig     &                           $ 0.9901 \pm 0.00072 $ &                           $ 0.8821 \pm 0.0043 $ &                           $ 0.8809 \pm 0.0043 $ \\
ff-RNN         &                           $ 0.9933 \pm 0.00046 $ &                           $ 0.8975 \pm 0.0041 $ &                           $ 0.8966 \pm 0.0042 $ \\
ff-RNNSig      &                           $ 0.9927 \pm 0.00015 $ &                           $ 0.8979 \pm 0.0038 $ &                           $ 0.8974 \pm 0.0039 $ \\
ff-Sig         &                           $ 0.9831 \pm 0.00045 $ &                           $ 0.8374 \pm 0.0042 $ &                            $ 0.8362 \pm 0.004 $ \\
ind-DeepSig    &  $  \mathbf{ \underline{ 0.9996 \pm 0.00013 }} $ &  $  \mathbf{ \underline{ 0.9807 \pm 0.0018 }} $ &  $  \mathbf{ \underline{ 0.9806 \pm 0.0019 }} $ \\
ind-RNN        &               $  \mathbf{ 0.9996 \pm 9.2e-05 } $ &            $  \underline{ 0.9727 \pm 0.0044 } $ &            $  \underline{ 0.9724 \pm 0.0045 } $ \\
ind-RNNSig     &                           $ 0.9975 \pm 0.00028 $ &                           $ 0.9349 \pm 0.0062 $ &                           $ 0.9341 \pm 0.0061 $ \\
ind-Sig        &                           $ 0.9941 \pm 0.00031 $ &                           $ 0.9059 \pm 0.0031 $ &                           $ 0.9049 \pm 0.0031 $ \\
lin-DeepSig    &                           $ 0.9946 \pm 0.00052 $ &                           $ 0.9157 \pm 0.0041 $ &                           $ 0.9145 \pm 0.0042 $ \\
lin-RNN        &                           $ 0.9949 \pm 0.00093 $ &                           $ 0.9161 \pm 0.0061 $ &                           $ 0.9149 \pm 0.0061 $ \\
lin-RNNSig     &                           $ 0.9945 \pm 0.00039 $ &                           $ 0.9026 \pm 0.0086 $ &                           $ 0.9014 \pm 0.0087 $ \\
lin-Sig        &                           $ 0.9896 \pm 0.00084 $ &                           $ 0.8725 \pm 0.0044 $ &                           $ 0.8714 \pm 0.0046 $ \\
zero-DeepSig   &                           $ 0.9939 \pm 0.00071 $ &                           $ 0.9112 \pm 0.0041 $ &                            $ 0.9101 \pm 0.004 $ \\
zero-RNN       &                           $ 0.9955 \pm 0.00031 $ &                           $ 0.9177 \pm 0.0028 $ &                            $ 0.9167 \pm 0.003 $ \\
zero-RNNSig    &                           $ 0.9932 \pm 0.00033 $ &                           $ 0.8954 \pm 0.0041 $ &                           $ 0.8946 \pm 0.0042 $ \\
zero-Sig       &                           $ 0.9854 \pm 0.00069 $ &                           $ 0.8427 \pm 0.0044 $ &                           $ 0.8419 \pm 0.0045 $ \\
\bottomrule
\end{tabular}

    \label{tab:pendigits}
    %}
\end{table}



\section{Discussion}

Our findings suggest that the choice of path imputation strategy can
\emph{drastically} affect the performance of signature-based models. We
observe this most prominently in `shallow' signature models. Among
signature models, we found that deep signature models~(\textsc{DeepSig})
are most robust in tackling irregular time series over different
imputations---comparable to non-signature RNNs, yet on average being
more parameter-efficient.

Overall, we find that uncertainty-aware approaches~(indicator
imputation and \textsc{GP-PoM}) are beneficial when imputing
irregularly-spaced time series for classification.
%
Crucially, uncertainty information has to be accessible during the
\emph{prediction step}. We find that this is indeed not the case for
the standard GP adapter~(despite the naming of `uncertainty-aware
framework'), since for each MC sample, the GP adapter model has no access
to missingness or uncertainty about the underlying imputation.
\textsc{GP-PoM}, our proposed end-to-end imputation strategy, shows
competitive classification performance, while considerably improving
upon the existing GP adapter. As for
limitations, \textsc{GP-PoM} sacrifices the GP adapter's ability to be
explicitly uncertain \emph{about} its own prediction~(due to the
variance of the MC sampled predictions), while the subsequent classifier
has to be able to handle the doubled feature dimensionality.

\section{Conclusion}

The signature transform has recently gained attention for being
a promising feature extractor that can be easily integrated to neural
networks. As we empirically demonstrated in this paper, the application
of signature transforms to real-world temporal data is fraught with
pitfalls---specifically, we found the choice of an imputation scheme to
be crucial for obtaining high predictive performance. Moreover, by integrating uncertainty to the prediction step, our proposed \textsc{GP-PoM} has demonstrated overall competitive performance and in particular improved robustness in signature models when dealing with irregularly-spaced and asynchronous time series. 


% \section{Experiments} %OLD
% % The flaws we have identified have been theoretically identified, and so we have been able to correct them theoretically as well. Nonetheless we have also performed experiments, testing different imputation schemes across several different signature-based models.
% For our empirical analysis, let us first introduce our experimental setup by specifying the investigated datasets, models, imputation schemes as well as prediction tasks at hand. Subsequently, we present and discuss our quantitative results.
% \subsection{Datasets}
% We make use of four real-world time series datasets: Physionet2012 challenge \cite{goldberger2000physiobank},
%     PenDigits \cite{Dua2019}, LSST \cite{allam2018photometric}, and CharacterTrajectories \cite{Dua2019}.

% \subsection{Models} 
% We study the following set of models:
% \begin{enumerate}
%     \item Sig, a simple signature model which involves a linear augmentation, the signature transform (signature block) and a final module of dense layers.
%     \item RNNSig, which extends the signature transform to a window-based stream of signatures, and where the final neural module is a GRU sliding over the stream of signatures.
%     %\item DeepSig, where two signature blocks are stacked.
%     \item RNN, a straight-forward RNN model using gated recurrent units (GRU) as RNN cells.
% \end{enumerate}
% For more details with respect to architectures and implemenation, please refer to the supplementary Sectoin XXXX.
% First, we examined the PhysioNet 2012 Mortality Prediction Dataset.

% We performed several experiments, testing different imputation schemes across several different signature models.
% To calculate the signature transform, we used the Signatory package of \cite{signatory}.

% \begin{figure*}[tbp]
%   \subcaptionbox{CharacterTrajectories-R}{%
%     \includegraphics[width=0.48\linewidth]{plots/barplot_main_CharacterTrajectories-MissingAtRandomSubsampler.pdf}\quad%
%     \includegraphics[width=0.48\linewidth]{plots/barplot_main_params_CharacterTrajectories-MissingAtRandomSubsampler.pdf}%
%   }\\%
%   \subcaptionbox{CharacterTrajectories-L}{%
%     \includegraphics[width=0.48\linewidth]{plots/barplot_main_CharacterTrajectories-LabelBasedSubsampler.pdf}\quad%
%     \includegraphics[width=0.48\linewidth]{plots/barplot_main_params_CharacterTrajectories-LabelBasedSubsampler.pdf}%
%   }\\%
%   \subcaptionbox{LSST-R}{%
%     \includegraphics[width=0.48\linewidth]{plots/barplot_main_LSST-MissingAtRandomSubsampler.pdf}\quad%
%     \includegraphics[width=0.48\linewidth]{plots/barplot_main_params_LSST-MissingAtRandomSubsampler.pdf}%
%   }\\%
%   \subcaptionbox{LSST-L}{%
%     \includegraphics[width=0.48\linewidth]{plots/barplot_main_LSST-LabelBasedSubsampler.pdf}\quad%
%     \includegraphics[width=0.48\linewidth]{plots/barplot_main_params_LSST-LabelBasedSubsampler.pdf}
%   }\\%
%   \subcaptionbox{PenDigits-R}{%
%      \includegraphics[width=0.48\linewidth]{plots/barplot_main_PenDigits-MissingAtRandomSubsampler.pdf}\quad%
%      \includegraphics[width=0.48\linewidth]{plots/barplot_main_params_PenDigits-MissingAtRandomSubsampler.pdf}
%   }\\%
%   \subcaptionbox{PenDigits-L}{%
%     \includegraphics[width=0.48\linewidth]{plots/barplot_main_PenDigits-LabelBasedSubsampler.pdf}\quad%
%     \includegraphics[width=0.48\linewidth]{plots/barplot_main_params_PenDigits-LabelBasedSubsampler.pdf}
%   }\\ %
%   \subcaptionbox{Physionet2012}{%
%     \includegraphics[width=0.48\linewidth]{plots/barplot_main_Physionet2012-.pdf}\quad%
%     \includegraphics[width=0.48\linewidth]{plots/barplot_main_params_Physionet2012-.pdf}
%   }
%   \caption{%
%     Experimental results.
%   }
%   \label{fig: barplots}
% \end{figure*}

% {\tiny
% \begin{table}[h]
%     \caption{\textbf{Physionet 2012}. The below methods are named as $<$imputation method$>-<$prediction model$>$. GP indicates an end-to-end trained Gaussian Process adapter, whereas PoM indicates our proposed extension, "posterior moments". The remaining imputation schemes have been applied as a preprocessing step, whereas the abbreviations "causal, ff, ind, lin, and zero" refer to: "causal, forward-filling, indicator, linear, and zero". Highlighting of the 3 best methods in decreasing order: bold as well as underlined, bold, and underlined.}
%     %\begin{tabular}{lll}
\toprule
metric &                                 AUROC &                     Average Precision \\
model            &                                       &                                       \\
\midrule
GP-DeepSig       &                           $ 0.83057 $ &                           $ 0.48053 $ \\
GP-DeepSig (PoM) &                            $ 0.8202 $ &                           $ 0.49333 $ \\
GP-RNN           &                           $ 0.82501 $ &                           $ 0.48862 $ \\
GP-RNN (PoM)     &                            $ 0.8411 $ &                           $ 0.50316 $ \\
GP-RNNSig        &                           $ 0.75912 $ &                           $ 0.36152 $ \\
GP-RNNSig (PoM)  &                           $ 0.77133 $ &                            $ 0.4031 $ \\
GP-Sig           &                           $ 0.73406 $ &                           $ 0.35228 $ \\
GP-Sig (PoM)     &                           $ 0.76132 $ &                           $ 0.34929 $ \\
causal-DeepSig   &                           $ 0.83786 $ &                           $ 0.49095 $ \\
causal-RNN       &               $  \mathbf{ 0.84687 } $ &               $  \mathbf{ 0.51729 } $ \\
causal-RNNSig    &                           $ 0.81896 $ &                           $ 0.46484 $ \\
causal-Sig       &                           $ 0.83908 $ &                           $ 0.48984 $ \\
ff-DeepSig       &                           $ 0.82189 $ &                           $ 0.48037 $ \\
ff-RNN           &            $  \underline{ 0.84655 } $ &  $  \mathbf{ \underline{ 0.52581 }} $ \\
ff-RNNSig        &                           $ 0.80963 $ &                           $ 0.43971 $ \\
ff-Sig           &                           $ 0.84356 $ &                            $ 0.5036 $ \\
ind-DeepSig      &                           $ 0.82432 $ &                           $ 0.48006 $ \\
ind-RNN          &  $  \mathbf{ \underline{ 0.84961 }} $ &            $  \underline{ 0.51611 } $ \\
ind-RNNSig       &                           $ 0.83766 $ &                           $ 0.50512 $ \\
ind-Sig          &                           $ 0.80753 $ &                           $ 0.46094 $ \\
lin-DeepSig      &                            $ 0.8338 $ &                           $ 0.49465 $ \\
lin-RNN          &                           $ 0.84303 $ &                           $ 0.51061 $ \\
lin-RNNSig       &                            $ 0.7769 $ &                           $ 0.42146 $ \\
lin-Sig          &                           $ 0.83442 $ &                           $ 0.49693 $ \\
zero-DeepSig     &                           $ 0.82405 $ &                           $ 0.47245 $ \\
zero-RNN         &                           $ 0.84041 $ &                           $ 0.49561 $ \\
zero-RNNSig      &                           $ 0.78151 $ &                           $ 0.41783 $ \\
zero-Sig         &                           $ 0.80593 $ &                           $ 0.44305 $ \\
\bottomrule
\end{tabular}
 %TODO: I need to update the result tables with repetitions including std estimates (curerntly hyperparameter search results - not repetitions.
%     \begin{tabular}{lll}
\toprule
metric &                                          AUROC &                              Average Precision \\
model          &                                                &                                                \\
\midrule
GP-PoM-DeepSig &                           $ 82.084 \pm 0.836 $ &                           $ 47.858 \pm 1.362 $ \\
GP-PoM-RNN     &                           $ 83.222 \pm 0.570 $ &                           $ 49.263 \pm 1.115 $ \\
GP-PoM-RNNSig  &                           $ 77.879 \pm 1.072 $ &                           $ 40.157 \pm 0.539 $ \\
GP-PoM-Sig     &                           $ 74.388 \pm 2.211 $ &                           $ 33.909 \pm 2.902 $ \\
GP-DeepSig     &                           $ 82.164 \pm 0.245 $ &                           $ 47.707 \pm 0.971 $ \\
GP-RNN         &                           $ 81.196 \pm 0.953 $ &                           $ 47.322 \pm 1.642 $ \\
GP-RNNSig      &                           $ 74.665 \pm 1.763 $ &                           $ 36.585 \pm 1.549 $ \\
GP-Sig         &                           $ 70.984 \pm 1.611 $ &                           $ 30.886 \pm 2.115 $ \\
causal-DeepSig &                           $ 83.487 \pm 0.574 $ &                           $ 48.924 \pm 0.935 $ \\
causal-RNN     &            $  \underline{ 84.689 \pm 0.325 } $ &  $  \mathbf{ \underline{ 52.646 \pm 0.460 }} $ \\
causal-RNNSig  &                           $ 82.074 \pm 0.080 $ &                           $ 47.691 \pm 0.214 $ \\
causal-Sig     &                           $ 83.499 \pm 0.597 $ &                           $ 47.809 \pm 0.783 $ \\
ff-DeepSig     &                           $ 82.766 \pm 0.646 $ &                           $ 47.971 \pm 1.562 $ \\
ff-RNN         &  $  \mathbf{ \underline{ 84.954 \pm 0.157 }} $ &               $  \mathbf{ 52.427 \pm 0.521 } $ \\
ff-RNNSig      &                           $ 80.916 \pm 0.607 $ &                           $ 44.979 \pm 1.347 $ \\
ff-Sig         &                           $ 84.328 \pm 0.213 $ &                           $ 51.043 \pm 0.547 $ \\
ind-DeepSig    &                           $ 82.332 \pm 0.467 $ &                           $ 47.150 \pm 1.020 $ \\
ind-RNN        &               $  \mathbf{ 84.906 \pm 0.211 } $ &            $  \underline{ 51.887 \pm 0.713 } $ \\
ind-RNNSig     &                           $ 83.651 \pm 0.199 $ &                           $ 49.872 \pm 0.570 $ \\
ind-Sig        &                           $ 81.570 \pm 0.473 $ &                           $ 45.807 \pm 1.318 $ \\
lin-DeepSig    &                           $ 83.168 \pm 0.650 $ &                           $ 48.937 \pm 1.291 $ \\
lin-RNN        &                           $ 84.367 \pm 0.176 $ &                           $ 50.431 \pm 0.340 $ \\
lin-RNNSig     &                           $ 77.336 \pm 0.469 $ &                           $ 41.888 \pm 0.583 $ \\
lin-Sig        &                           $ 83.354 \pm 0.223 $ &                           $ 48.900 \pm 0.480 $ \\
zero-DeepSig   &                           $ 80.101 \pm 2.126 $ &                           $ 44.340 \pm 3.076 $ \\
zero-RNN       &                           $ 83.571 \pm 0.227 $ &                           $ 49.304 \pm 0.597 $ \\
zero-RNNSig    &                           $ 76.246 \pm 1.436 $ &                           $ 39.502 \pm 2.232 $ \\
zero-Sig       &                           $ 80.645 \pm 0.097 $ &                           $ 44.728 \pm 0.264 $ \\
\bottomrule
\end{tabular}

% \end{table}
% }



% \section{Recommendations} %OLD
% We finish with some recommendations for the practitioner, based on our findings.

% Whether causality is a concern or not, then causal signature imputation is simple to implement and effective at correcting the identified issues. If one takes the point of view that the signature transform is defined on data (rather than paths), then all that is necessary is to pick one's favourite imputation scheme, apply it as usual, and then apply the $\Phi$ of equation \eqref{eq:phi} to the imputed data, before using it. In more complicated models such as those proposed by \cite{kidger2019deep}, then a perfectly valid `zero thought' approach is to precompose $\Phi$ of equation \eqref{eq:phi} to every usage of the signature transform.

% If causality is not a concern, then we recommend the variation of a Gaussian process adapter proposed in section \ref{section:ourgpadapter}. The implementation of this is much less straightforward, however. It remains a reasonable compromise to use linear data-imputation, which is unique in avoiding the flaw of dependence on unrelated channels, whilst accepting that there may be slight inconsistencies at either end of a partially observed time series.

% We do not recommend applying any standard imputation scheme unless it has been modified by causal signature imputation.

% \section{Conclusion} %OLD
% When using the signature transform in machine learning, we have identified a novel issue in the greater-than-usual importance of the choice of imputation strategy.

% We have identified two key flaws that can arise when mishandling this issue, and show that they are necessarily exhibited by all conventional imputation strategies not specifically adapted for use with the signature transform.

% We have then proposed two practical, novel, solutions to overcome them,
% namely causal signature imputation and a Gaussian process adapter, the
% latter of which is of independent interest in other scenarios due to the
% way it incorporates uncertainty information about the \emph{predictions}
% themselves.

% As a result, we are able to summarise with `recommendations for the practitioner' on the appropriateness of different techniques.

\section*{Broader Impact}

Whilst the task of converting observed data into a path in data space is particularly important for signatures, it also arises in the context of, for example, convolutional and recurrent neural networks.

Convolutions are often thought of in terms of discrete sums, but they are perhaps more naturally described as the integral cross-correlation between the underlying data path $f$ and the learnt filter $g_\theta$. Given sample points $t_1, \ldots, t_n \in [0, T]$, this integral is then approximated via numerical quadrature:
\begin{equation*}
    \frac{1}{T}\int_0^T f(t) g_\theta(t) \mathrm{d}t \approx \frac{1}{n}\sum_{i = 1}^n f(t_i) g_\theta(t_i),
\end{equation*}
although the $1/n$ scaling is really only justified in the case that the $t_i$ are equally spaced.\footnote{The $g_\theta$ is typically a step function in `normal' convolutional layers. Some works exists on replacing it with e.g. B-splines \cite{fey2018splinecnn} to better handle irregular data. The oddity of scaling by $1/n$ with irregular data has not been explicitly addressed in the literature, at least to our knowledge; indeed quite conversely we have seen it used without remark.} Thus we see that with convolutions, we are implicitly interpreting the observed data as a path in data space.

Similarly, the connection between dynamical systems and recurrent neural networks are well known \cite{FUNAHASHI1993801, continuousrnn}, and these tend to use a similar setup.

For non-signature methods as for signature methods, this implicit usage of data as a path in data space often seems to be swept under the rug, and we have demonstrated that this is deserving further attention.
% While our focus here is specifically to study path constructions for signatures, we consider implicit path constructions as the manifestation of a larger problem which deserves more attention.

\begin{ack}
M.M, M.H, and C.B were supported by the SNSF Starting Grant `Significant
Pattern Mining'. The authors are grateful to Patrick Kidger for valuable discussions, guidance, and code contributions.
\end{ack}

\bibliography{ref}
\bibliographystyle{apalike} %ieeetr
\newpage
\appendix

\section{Appendix}\label{sec:Appendix}

\subsection{Further Experiments}\label{supp:Experiments}

\begin{table}[htbp]
    \begin{center}
	\caption{Character Trajectories, random subsampling}
	%\small{
	\begin{tabular}{llll}
\toprule
metric &                                          w-AUROC &                                            BAC &                                     Accuracy \\
model          &                                                  &                                                &                                              \\
\midrule
GP-DeepSig     &                             $ 0.9899 \pm 0.011 $ &                           $ 0.9082 \pm 0.024 $ &                         $ 0.9047 \pm 0.023 $ \\
GP-PoM-DeepSig &                             $ 0.997 \pm 0.0039 $ &                            $ 0.9401 \pm 0.05 $ &                         $ 0.9364 \pm 0.053 $ \\
GP-PoM-RNN     &            $  \underline{ 0.9997 \pm 0.00011 } $ &                          $ 0.9801 \pm 0.0051 $ &                        $ 0.9811 \pm 0.0051 $ \\
GP-PoM-RNNSig  &                           $ 0.9979 \pm 0.00074 $ &                          $ 0.9331 \pm 0.0096 $ &                         $ 0.9384 \pm 0.009 $ \\
GP-PoM-Sig     &                           $ 0.9958 \pm 0.00031 $ &                          $ 0.8957 \pm 0.0094 $ &                        $ 0.8993 \pm 0.0091 $ \\
GP-RNN         &                           $ 0.9991 \pm 0.00032 $ &                          $ 0.9628 \pm 0.0069 $ &                        $ 0.9649 \pm 0.0071 $ \\
GP-RNNSig      &                            $ 0.994 \pm 0.00094 $ &                           $ 0.8759 \pm 0.021 $ &                          $ 0.8814 \pm 0.02 $ \\
GP-Sig         &                             $ 0.9486 \pm 0.018 $ &                           $ 0.6128 \pm 0.064 $ &                         $ 0.6245 \pm 0.065 $ \\
causal-DeepSig &                           $ 0.9996 \pm 0.00023 $ &                          $ 0.9777 \pm 0.0023 $ &                        $ 0.9795 \pm 0.0018 $ \\
causal-RNN     &                           $ 0.9995 \pm 0.00023 $ &                          $ 0.9766 \pm 0.0072 $ &                        $ 0.9781 \pm 0.0068 $ \\
causal-RNNSig  &                           $ 0.9981 \pm 0.00044 $ &                          $ 0.9327 \pm 0.0073 $ &                        $ 0.9375 \pm 0.0074 $ \\
causal-Sig     &                            $ 0.9674 \pm 0.0058 $ &                           $ 0.7139 \pm 0.038 $ &                         $ 0.7325 \pm 0.036 $ \\
ff-DeepSig     &                            $ 0.9996 \pm 0.0003 $ &                          $ 0.9797 \pm 0.0038 $ &                        $ 0.9812 \pm 0.0037 $ \\
ff-RNN         &                            $ 0.9995 \pm 0.0001 $ &                          $ 0.9779 \pm 0.0031 $ &                        $ 0.9794 \pm 0.0028 $ \\
ff-RNNSig      &                           $ 0.9984 \pm 0.00047 $ &                          $ 0.9411 \pm 0.0077 $ &                         $ 0.946 \pm 0.0074 $ \\
ff-Sig         &                             $ 0.5431 \pm 0.042 $ &                           $ 0.07387 \pm 0.03 $ &                        $ 0.07117 \pm 0.024 $ \\
ind-DeepSig    &                           $ 0.9996 \pm 0.00033 $ &               $  \mathbf{ 0.9863 \pm 0.005 } $ &            $  \mathbf{ 0.9873 \pm 0.0048 } $ \\
ind-RNN        &                           $ 0.9995 \pm 0.00024 $ &                           $ 0.975 \pm 0.0053 $ &                         $ 0.9766 \pm 0.005 $ \\
ind-RNNSig     &                           $ 0.9975 \pm 0.00078 $ &                           $ 0.9309 \pm 0.011 $ &                        $ 0.9364 \pm 0.0095 $ \\
ind-Sig        &                              $ 0.6692 \pm 0.18 $ &                            $ 0.1848 \pm 0.19 $ &                          $ 0.1907 \pm 0.19 $ \\
lin-DeepSig    &  $  \mathbf{ \underline{ 0.9998 \pm 6.8e-05 }} $ &  $  \mathbf{ \underline{ 0.989 \pm 0.0021 }} $ &  $  \mathbf{ \underline{ 0.99 \pm 0.002 }} $ \\
lin-RNN        &                           $ 0.9993 \pm 0.00043 $ &                           $ 0.9767 \pm 0.009 $ &                         $ 0.9779 \pm 0.008 $ \\
lin-RNNSig     &                           $ 0.9977 \pm 0.00037 $ &                          $ 0.9275 \pm 0.0066 $ &                        $ 0.9327 \pm 0.0066 $ \\
lin-Sig        &                             $ 0.5502 \pm 0.067 $ &                          $ 0.09436 \pm 0.033 $ &                        $ 0.09958 \pm 0.041 $ \\
zero-DeepSig   &               $  \mathbf{ 0.9998 \pm 0.00013 } $ &           $  \underline{ 0.9834 \pm 0.0064 } $ &         $  \underline{ 0.9845 \pm 0.0062 } $ \\
zero-RNN       &                           $ 0.9989 \pm 0.00052 $ &                             $ 0.96 \pm 0.011 $ &                          $ 0.9625 \pm 0.01 $ \\
zero-RNNSig    &                           $ 0.9968 \pm 0.00063 $ &                          $ 0.9215 \pm 0.0088 $ &                        $ 0.9274 \pm 0.0082 $ \\
zero-Sig       &                              $ 0.97 \pm 0.0039 $ &                           $ 0.6996 \pm 0.042 $ &                           $ 0.717 \pm 0.04 $ \\
\bottomrule
\end{tabular}

	\label{supp: tab character}
	%}
	\end{center}
\end{table}

\begin{table}[htbp]
    \begin{center}
	\caption{PenDigits, random subsampling}
	%\small{
	\begin{tabular}{llll}
\toprule
metric &                                        w-AUROC &                                            BAC &                                       Accuracy \\
model          &                                                &                                                &                                                \\
\midrule
GP-PoM-DeepSig &                           $ 99.515 \pm 0.078 $ &                           $ 92.151 \pm 0.555 $ &                           $ 92.098 \pm 0.548 $ \\
GP-PoM-RNN     &                           $ 99.564 \pm 0.072 $ &            $  \underline{ 92.757 \pm 0.735 } $ &            $  \underline{ 92.699 \pm 0.733 } $ \\
GP-PoM-RNNSig  &                           $ 98.967 \pm 0.253 $ &                           $ 88.148 \pm 1.588 $ &                           $ 88.113 \pm 1.579 $ \\
GP-PoM-Sig     &                           $ 99.028 \pm 0.099 $ &                           $ 87.352 \pm 0.898 $ &                           $ 87.290 \pm 0.903 $ \\
\midrule
GP-DeepSig     &                           $ 90.509 \pm 0.164 $ &                           $ 54.545 \pm 0.426 $ &                           $ 54.513 \pm 0.451 $ \\
GP-RNN         &                           $ 91.961 \pm 0.856 $ &                           $ 57.930 \pm 2.079 $ &                           $ 57.900 \pm 2.088 $ \\
GP-RNNSig      &                           $ 86.740 \pm 0.585 $ &                           $ 46.842 \pm 1.255 $ &                           $ 46.867 \pm 1.218 $ \\
GP-Sig         &                           $ 83.511 \pm 0.485 $ &                           $ 41.747 \pm 0.428 $ &                           $ 41.809 \pm 0.425 $ \\
\midrule
causal-DeepSig &                           $ 99.096 \pm 0.116 $ &                           $ 89.480 \pm 0.359 $ &                           $ 89.434 \pm 0.362 $ \\
causal-RNN     &                           $ 99.288 \pm 0.066 $ &                           $ 89.526 \pm 0.535 $ &                           $ 89.474 \pm 0.539 $ \\
causal-RNNSig  &                           $ 99.165 \pm 0.067 $ &                           $ 88.807 \pm 0.613 $ &                           $ 88.759 \pm 0.617 $ \\
causal-Sig     &                           $ 97.870 \pm 0.224 $ &                           $ 80.065 \pm 0.980 $ &                           $ 80.011 \pm 0.971 $ \\
\midrule
ff-DeepSig     &                           $ 99.141 \pm 0.068 $ &                           $ 88.974 \pm 0.656 $ &                           $ 88.902 \pm 0.644 $ \\
ff-RNN         &                           $ 99.311 \pm 0.067 $ &                           $ 90.067 \pm 0.247 $ &                           $ 90.029 \pm 0.247 $ \\
ff-RNNSig      &                           $ 99.203 \pm 0.063 $ &                           $ 88.930 \pm 0.513 $ &                           $ 88.902 \pm 0.528 $ \\
ff-Sig         &                           $ 98.425 \pm 0.069 $ &                           $ 84.458 \pm 0.468 $ &                           $ 84.374 \pm 0.477 $ \\
\midrule
ind-DeepSig    &               $  \mathbf{ 99.607 \pm 0.059 } $ &               $  \mathbf{ 93.156 \pm 0.738 } $ &               $  \mathbf{ 93.087 \pm 0.751 } $ \\
ind-RNN        &  $  \mathbf{ \underline{ 99.733 \pm 0.044 }} $ &  $  \mathbf{ \underline{ 94.124 \pm 0.412 }} $ &  $  \mathbf{ \underline{ 94.071 \pm 0.415 }} $ \\
ind-RNNSig     &                           $ 99.549 \pm 0.041 $ &                           $ 91.604 \pm 0.278 $ &                           $ 91.532 \pm 0.268 $ \\
ind-Sig        &                           $ 98.708 \pm 0.040 $ &                           $ 84.544 \pm 0.538 $ &                           $ 84.505 \pm 0.563 $ \\
\midrule
lin-DeepSig    &                           $ 99.407 \pm 0.151 $ &                           $ 91.418 \pm 1.075 $ &                           $ 91.366 \pm 1.086 $ \\
lin-RNN        &                           $ 99.510 \pm 0.041 $ &                           $ 91.862 \pm 0.582 $ &                           $ 91.812 \pm 0.594 $ \\
lin-RNNSig     &            $  \underline{ 99.591 \pm 0.036 } $ &                           $ 91.556 \pm 0.518 $ &                           $ 91.521 \pm 0.539 $ \\
lin-Sig        &                           $ 99.029 \pm 0.094 $ &                           $ 87.116 \pm 0.612 $ &                           $ 87.038 \pm 0.612 $ \\
\midrule
zero-DeepSig   &                           $ 99.334 \pm 0.077 $ &                           $ 89.774 \pm 0.541 $ &                           $ 89.686 \pm 0.553 $ \\
zero-RNN       &                           $ 99.403 \pm 0.112 $ &                           $ 90.729 \pm 0.618 $ &                           $ 90.698 \pm 0.620 $ \\
zero-RNNSig    &                           $ 99.150 \pm 0.046 $ &                           $ 87.948 \pm 0.248 $ &                           $ 87.879 \pm 0.243 $ \\
zero-Sig       &                           $ 98.623 \pm 0.073 $ &                           $ 83.935 \pm 0.382 $ &                           $ 83.905 \pm 0.375 $ \\
\bottomrule
\end{tabular}

	\label{supp: tab pendigits}
	%}
	\end{center}
\end{table}

\begin{table}[htbp]
    \begin{center}
	\caption{LSST, label-based subsampling}
	%\small{
	\begin{tabular}{llll}
\toprule
metric &                                         w-AUROC &                                            BAC &                                       Accuracy \\
model          &                                                 &                                                &                                                \\
\midrule
GP-DeepSig     &                            $ 0.5969 \pm 0.034 $ &                           $ 0.1314 \pm 0.039 $ &                           $ 0.2833 \pm 0.055 $ \\
GP-PoM-DeepSig &                            $ 0.6441 \pm 0.068 $ &                           $ 0.1386 \pm 0.024 $ &                          $ 0.08743 \pm 0.029 $ \\
GP-PoM-RNN     &                            $ 0.8281 \pm 0.043 $ &                           $ 0.2948 \pm 0.071 $ &                            $ 0.2978 \pm 0.12 $ \\
GP-PoM-RNNSig  &                            $ 0.8293 \pm 0.015 $ &                           $ 0.3175 \pm 0.023 $ &                           $ 0.3544 \pm 0.036 $ \\
GP-PoM-Sig     &                            $ 0.5882 \pm 0.014 $ &                          $ 0.1378 \pm 0.0053 $ &                           $ 0.1019 \pm 0.014 $ \\
GP-RNN         &                            $ 0.6927 \pm 0.015 $ &                          $ 0.1656 \pm 0.0046 $ &                           $ 0.339 \pm 0.0036 $ \\
GP-RNNSig      &                           $ 0.6003 \pm 0.0048 $ &                           $ 0.1678 \pm 0.016 $ &                          $ 0.3363 \pm 0.0021 $ \\
GP-Sig         &                           $ 0.5738 \pm 0.0055 $ &                           $ 0.1406 \pm 0.012 $ &                           $ 0.2429 \pm 0.021 $ \\
causal-DeepSig &                            $ 0.7534 \pm 0.035 $ &                           $ 0.2548 \pm 0.059 $ &                            $ 0.394 \pm 0.058 $ \\
causal-RNN     &                           $ 0.8411 \pm 0.0061 $ &                           $ 0.3776 \pm 0.017 $ &                           $ 0.5301 \pm 0.017 $ \\
causal-RNNSig  &                           $ 0.8234 \pm 0.0023 $ &                           $ 0.3357 \pm 0.028 $ &                          $ 0.5028 \pm 0.0097 $ \\
causal-Sig     &                            $ 0.5884 \pm 0.032 $ &                           $ 0.1226 \pm 0.028 $ &                           $ 0.3416 \pm 0.013 $ \\
ff-DeepSig     &                            $ 0.7776 \pm 0.026 $ &                           $ 0.2747 \pm 0.042 $ &                           $ 0.4224 \pm 0.054 $ \\
ff-RNN         &                           $ 0.8415 \pm 0.0067 $ &                           $ 0.3862 \pm 0.016 $ &                          $ 0.5298 \pm 0.0067 $ \\
ff-RNNSig      &                           $ 0.8243 \pm 0.0044 $ &                           $ 0.3408 \pm 0.014 $ &                          $ 0.5056 \pm 0.0068 $ \\
ff-Sig         &                            $ 0.642 \pm 0.0081 $ &                           $ 0.1478 \pm 0.012 $ &                          $ 0.3526 \pm 0.0052 $ \\
ind-DeepSig    &                 $  \mathbf{ 0.9535 \pm 0.01 } $ &               $  \mathbf{ 0.5212 \pm 0.021 } $ &               $  \mathbf{ 0.7728 \pm 0.022 } $ \\
ind-RNN        &  $  \mathbf{ \underline{ 0.9813 \pm 0.0024 }} $ &  $  \mathbf{ \underline{ 0.6189 \pm 0.039 }} $ &  $  \mathbf{ \underline{ 0.8361 \pm 0.015 }} $ \\
ind-RNNSig     &                            $ 0.8263 \pm 0.018 $ &                           $ 0.2912 \pm 0.031 $ &                           $ 0.4115 \pm 0.025 $ \\
ind-Sig        &                           $ 0.5708 \pm 0.0086 $ &                          $ 0.1281 \pm 0.0095 $ &                           $ 0.3262 \pm 0.011 $ \\
lin-DeepSig    &                            $ 0.7396 \pm 0.032 $ &                          $ 0.2136 \pm 0.0075 $ &                            $ 0.365 \pm 0.063 $ \\
lin-RNN        &                            $ 0.8493 \pm 0.003 $ &                           $ 0.401 \pm 0.0097 $ &           $  \underline{ 0.5402 \pm 0.0082 } $ \\
lin-RNNSig     &                            $ 0.8288 \pm 0.008 $ &                           $ 0.3255 \pm 0.013 $ &                           $ 0.4933 \pm 0.024 $ \\
lin-Sig        &                            $ 0.6669 \pm 0.018 $ &                          $ 0.1734 \pm 0.0078 $ &                          $ 0.3573 \pm 0.0057 $ \\
zero-DeepSig   &                            $ 0.7578 \pm 0.012 $ &                            $ 0.236 \pm 0.033 $ &                           $ 0.4202 \pm 0.017 $ \\
zero-RNN       &            $  \underline{ 0.8791 \pm 0.0095 } $ &            $  \underline{ 0.4048 \pm 0.019 } $ &                           $ 0.5327 \pm 0.021 $ \\
zero-RNNSig    &                            $ 0.7943 \pm 0.005 $ &                           $ 0.3438 \pm 0.012 $ &                          $ 0.4652 \pm 0.0045 $ \\
zero-Sig       &                            $ 0.5297 \pm 0.011 $ &                           $ 0.1195 \pm 0.011 $ &                           $ 0.2853 \pm 0.084 $ \\
\bottomrule
\end{tabular}

	\label{supp: tab lsst label-based}
	%}
	\end{center}
\end{table}

\begin{table}[htbp]
    \begin{center}
	\caption{LSST, random subsampling}
	%\small{
	\begin{tabular}{llll}
\toprule
metric &                                         w-AUROC &                                             BAC &                                       Accuracy \\
model          &                                                 &                                                 &                                                \\
\midrule
GP-DeepSig     &                           $ 0.6213 \pm 0.0096 $ &                            $ 0.1619 \pm 0.018 $ &                          $ 0.3432 \pm 0.0016 $ \\
GP-PoM-DeepSig &                            $ 0.7077 \pm 0.029 $ &                            $ 0.2182 \pm 0.027 $ &                           $ 0.4011 \pm 0.024 $ \\
GP-PoM-RNN     &                           $ 0.8232 \pm 0.0083 $ &             $  \underline{ 0.3737 \pm 0.019 } $ &                           $ 0.5251 \pm 0.026 $ \\
GP-PoM-RNNSig  &                            $ 0.7584 \pm 0.016 $ &                            $ 0.2668 \pm 0.019 $ &                           $ 0.4179 \pm 0.019 $ \\
GP-PoM-Sig     &                            $ 0.588 \pm 0.0056 $ &                             $ 0.132 \pm 0.013 $ &                           $ 0.3461 \pm 0.008 $ \\
GP-RNN         &                             $ 0.6264 \pm 0.03 $ &                            $ 0.1736 \pm 0.022 $ &                           $ 0.3452 \pm 0.012 $ \\
GP-RNNSig      &                            $ 0.6059 \pm 0.004 $ &                            $ 0.1714 \pm 0.015 $ &                          $ 0.3371 \pm 0.0065 $ \\
GP-Sig         &                            $ 0.5798 \pm 0.013 $ &                            $ 0.1361 \pm 0.011 $ &                            $ 0.331 \pm 0.004 $ \\
causal-DeepSig &                            $ 0.7427 \pm 0.021 $ &                            $ 0.2218 \pm 0.027 $ &                            $ 0.341 \pm 0.066 $ \\
causal-RNN     &               $  \mathbf{ 0.8394 \pm 0.0073 } $ &                $  \mathbf{ 0.3741 \pm 0.023 } $ &  $  \mathbf{ \underline{ 0.5456 \pm 0.011 }} $ \\
causal-RNNSig  &                             $ 0.7719 \pm 0.05 $ &                            $ 0.3168 \pm 0.055 $ &                            $ 0.464 \pm 0.053 $ \\
causal-Sig     &                           $ 0.5268 \pm 0.0082 $ &                           $ 0.1163 \pm 0.0081 $ &                           $ 0.2603 \pm 0.099 $ \\
ff-DeepSig     &                            $ 0.7876 \pm 0.012 $ &                            $ 0.2787 \pm 0.022 $ &                           $ 0.4685 \pm 0.014 $ \\
ff-RNN         &  $  \mathbf{ \underline{ 0.8427 \pm 0.0057 }} $ &  $  \mathbf{ \underline{ 0.3832 \pm 0.0055 }} $ &            $  \underline{ 0.5324 \pm 0.013 } $ \\
ff-RNNSig      &                           $ 0.8229 \pm 0.0035 $ &                            $ 0.3352 \pm 0.011 $ &                            $ 0.502 \pm 0.011 $ \\
ff-Sig         &                            $ 0.5603 \pm 0.023 $ &                            $ 0.1298 \pm 0.015 $ &                            $ 0.2852 \pm 0.12 $ \\
ind-DeepSig    &                            $ 0.6986 \pm 0.016 $ &                            $ 0.2081 \pm 0.012 $ &                             $ 0.364 \pm 0.04 $ \\
ind-RNN        &                             $ 0.7696 \pm 0.03 $ &                            $ 0.2918 \pm 0.026 $ &                           $ 0.4218 \pm 0.044 $ \\
ind-RNNSig     &                             $ 0.637 \pm 0.015 $ &                            $ 0.1837 \pm 0.016 $ &                           $ 0.3031 \pm 0.019 $ \\
ind-Sig        &                            $ 0.5383 \pm 0.015 $ &                            $ 0.1221 \pm 0.023 $ &                          $ 0.3267 \pm 0.0036 $ \\
lin-DeepSig    &                             $ 0.7516 \pm 0.04 $ &                            $ 0.2366 \pm 0.044 $ &                           $ 0.4032 \pm 0.054 $ \\
lin-RNN        &            $  \underline{ 0.8378 \pm 0.0051 } $ &                            $ 0.3682 \pm 0.024 $ &              $  \mathbf{ 0.5344 \pm 0.0061 } $ \\
lin-RNNSig     &                             $ 0.8159 \pm 0.01 $ &                             $ 0.2981 \pm 0.02 $ &                           $ 0.4929 \pm 0.016 $ \\
lin-Sig        &                             $ 0.655 \pm 0.031 $ &                            $ 0.1711 \pm 0.035 $ &                          $ 0.3633 \pm 0.0098 $ \\
zero-DeepSig   &                            $ 0.6951 \pm 0.049 $ &                            $ 0.1523 \pm 0.047 $ &                           $ 0.3527 \pm 0.041 $ \\
zero-RNN       &                           $ 0.8174 \pm 0.0042 $ &                            $ 0.3558 \pm 0.027 $ &                           $ 0.5007 \pm 0.022 $ \\
zero-RNNSig    &                            $ 0.776 \pm 0.0063 $ &                            $ 0.3129 \pm 0.016 $ &                           $ 0.4595 \pm 0.013 $ \\
zero-Sig       &                             $ 0.529 \pm 0.015 $ &                            $ 0.1187 \pm 0.023 $ &                          $ 0.3261 \pm 0.0065 $ \\
\bottomrule
\end{tabular}

	\label{supp: tab lsst random}
	%}
	\end{center}
\end{table}

% \begin{table}[htbp]
%     \begin{center}
% 	\caption{Character Trajectories }
% 	\small{
% 	\begin{tabular}{lllll}
\toprule
       & metric &                                          w-AUROC &                                             BAC &                                        Accuracy \\
subsampling & model &                                                  &                                                 &                                                 \\
\midrule
Label-based & GP-DeepSig &                             $ 0.9929 \pm 0.007 $ &                             $ 0.8955 \pm 0.03 $ &                            $ 0.8937 \pm 0.031 $ \\
       & GP-RNN &                           $ 0.9997 \pm 0.00011 $ &                           $ 0.9771 \pm 0.0027 $ &                           $ 0.9787 \pm 0.0025 $ \\
       & GP-RNNSig &                             $ 0.9667 \pm 0.024 $ &                             $ 0.6572 \pm 0.14 $ &                             $ 0.6705 \pm 0.13 $ \\
       & GP-Sig &                             $ 0.9528 \pm 0.016 $ &                            $ 0.6242 \pm 0.061 $ &                             $ 0.6361 \pm 0.06 $ \\
       & GP-PoM-DeepSig &                            $ 0.9958 \pm 0.0067 $ &                            $ 0.9516 \pm 0.015 $ &                            $ 0.9496 \pm 0.017 $ \\
       & GP-PoM-RNN &            $  \underline{ 0.9997 \pm 0.00015 } $ &               $  \mathbf{ 0.9816 \pm 0.0066 } $ &                $  \mathbf{ 0.9827 \pm 0.006 } $ \\
       & GP-PoM-RNNSig &                            $ 0.997 \pm 0.00089 $ &                            $ 0.9278 \pm 0.012 $ &                            $ 0.9323 \pm 0.011 $ \\
       & GP-PoM-Sig &                           $ 0.9952 \pm 0.00075 $ &                            $ 0.8863 \pm 0.014 $ &                            $ 0.8901 \pm 0.013 $ \\
       & causal-DeepSig &                           $ 0.9994 \pm 0.00024 $ &                           $ 0.9727 \pm 0.0071 $ &                           $ 0.9744 \pm 0.0062 $ \\
       & causal-RNN &                           $ 0.9996 \pm 9.8e-05 $ &                           $ 0.9724 \pm 0.0052 $ &                           $ 0.9741 \pm 0.0048 $ \\
       & causal-RNNSig &                            $ 0.9952 \pm 0.0016 $ &                            $ 0.8992 \pm 0.023 $ &                            $ 0.9058 \pm 0.022 $ \\
       & causal-Sig &                              $ 0.9575 \pm 0.05 $ &                             $ 0.6631 \pm 0.22 $ &                             $ 0.6826 \pm 0.21 $ \\
       & ff-DeepSig &                           $ 0.9995 \pm 0.00041 $ &                           $ 0.9796 \pm 0.0068 $ &                           $ 0.9808 \pm 0.0066 $ \\
       & ff-RNN &                           $ 0.9994 \pm 0.00011 $ &                           $ 0.9694 \pm 0.0049 $ &                           $ 0.9716 \pm 0.0044 $ \\
       & ff-RNNSig &                           $ 0.9972 \pm 0.00071 $ &                            $ 0.9257 \pm 0.011 $ &                             $ 0.9315 \pm 0.01 $ \\
       & ff-Sig &                             $ 0.9483 \pm 0.081 $ &                             $ 0.6717 \pm 0.26 $ &                             $ 0.6865 \pm 0.26 $ \\
       & ind-DeepSig &  $  \mathbf{ \underline{ 0.9999 \pm 0.00013 }} $ &  $  \mathbf{ \underline{ 0.9859 \pm 0.0029 }} $ &  $  \mathbf{ \underline{ 0.9872 \pm 0.0026 }} $ \\
       & ind-RNN &                            $ 0.9992 \pm 0.0002 $ &                           $ 0.9641 \pm 0.0041 $ &                           $ 0.9667 \pm 0.0037 $ \\
       & ind-RNNSig &                            $ 0.998 \pm 0.00032 $ &                           $ 0.9379 \pm 0.0046 $ &                           $ 0.9423 \pm 0.0044 $ \\
       & ind-Sig &                               $ 0.9166 \pm 0.1 $ &                             $ 0.5642 \pm 0.23 $ &                             $ 0.5838 \pm 0.23 $ \\
       & lin-DeepSig &                            $ 0.9997 \pm 0.0001 $ &            $  \underline{ 0.9805 \pm 0.0074 } $ &            $  \underline{ 0.9822 \pm 0.0067 } $ \\
       & lin-RNN &                           $ 0.9988 \pm 0.00059 $ &                            $ 0.9691 \pm 0.013 $ &                            $ 0.9712 \pm 0.012 $ \\
       & lin-RNNSig &                           $ 0.9988 \pm 0.00035 $ &                           $ 0.9485 \pm 0.0092 $ &                           $ 0.9529 \pm 0.0084 $ \\
       & lin-Sig &                              $ 0.8044 \pm 0.18 $ &                             $ 0.3119 \pm 0.24 $ &                             $ 0.3233 \pm 0.25 $ \\
       & zero-DeepSig &                $  \mathbf{ 0.9998 \pm 0.0001 } $ &                           $ 0.9803 \pm 0.0036 $ &                           $ 0.9819 \pm 0.0036 $ \\
       & zero-RNN &                           $ 0.9997 \pm 0.00014 $ &                           $ 0.9743 \pm 0.0057 $ &                            $ 0.9755 \pm 0.006 $ \\
       & zero-RNNSig &                             $ 0.997 \pm 0.0013 $ &                            $ 0.9175 \pm 0.018 $ &                            $ 0.9237 \pm 0.017 $ \\
       & zero-Sig &                              $ 0.7773 \pm 0.24 $ &                             $ 0.3799 \pm 0.34 $ &                             $ 0.3896 \pm 0.35 $ \\
Random & GP-DeepSig &                             $ 0.9899 \pm 0.011 $ &                            $ 0.9082 \pm 0.024 $ &                            $ 0.9047 \pm 0.023 $ \\
       & GP-RNN &                           $ 0.9991 \pm 0.00032 $ &                           $ 0.9628 \pm 0.0069 $ &                           $ 0.9649 \pm 0.0071 $ \\
       & GP-RNNSig &                            $ 0.994 \pm 0.00094 $ &                            $ 0.8759 \pm 0.021 $ &                             $ 0.8814 \pm 0.02 $ \\
       & GP-Sig &                             $ 0.9486 \pm 0.018 $ &                            $ 0.6128 \pm 0.064 $ &                            $ 0.6245 \pm 0.065 $ \\
       & GP-PoM-DeepSig &                             $ 0.997 \pm 0.0039 $ &                             $ 0.9401 \pm 0.05 $ &                            $ 0.9364 \pm 0.053 $ \\
       & GP-PoM-RNN &            $  \underline{ 0.9997 \pm 0.00011 } $ &                           $ 0.9801 \pm 0.0051 $ &                           $ 0.9811 \pm 0.0051 $ \\
       & GP-PoM-RNNSig &                           $ 0.9979 \pm 0.00074 $ &                           $ 0.9331 \pm 0.0096 $ &                            $ 0.9384 \pm 0.009 $ \\
       & GP-PoM-Sig &                           $ 0.9958 \pm 0.00031 $ &                           $ 0.8957 \pm 0.0094 $ &                           $ 0.8993 \pm 0.0091 $ \\
       & causal-DeepSig &                           $ 0.9996 \pm 0.00023 $ &                           $ 0.9777 \pm 0.0023 $ &                           $ 0.9795 \pm 0.0018 $ \\
       & causal-RNN &                           $ 0.9995 \pm 0.00023 $ &                           $ 0.9766 \pm 0.0072 $ &                           $ 0.9781 \pm 0.0068 $ \\
       & causal-RNNSig &                           $ 0.9981 \pm 0.00044 $ &                           $ 0.9327 \pm 0.0073 $ &                           $ 0.9375 \pm 0.0074 $ \\
       & causal-Sig &                            $ 0.9674 \pm 0.0058 $ &                            $ 0.7139 \pm 0.038 $ &                            $ 0.7325 \pm 0.036 $ \\
       & ff-DeepSig &                            $ 0.9996 \pm 0.0003 $ &                           $ 0.9797 \pm 0.0038 $ &                           $ 0.9812 \pm 0.0037 $ \\
       & ff-RNN &                            $ 0.9995 \pm 0.0001 $ &                           $ 0.9779 \pm 0.0031 $ &                           $ 0.9794 \pm 0.0028 $ \\
       & ff-RNNSig &                           $ 0.9984 \pm 0.00047 $ &                           $ 0.9411 \pm 0.0077 $ &                            $ 0.946 \pm 0.0074 $ \\
       & ff-Sig &                             $ 0.5431 \pm 0.042 $ &                            $ 0.07387 \pm 0.03 $ &                           $ 0.07117 \pm 0.024 $ \\
       & ind-DeepSig &                           $ 0.9996 \pm 0.00033 $ &                $  \mathbf{ 0.9863 \pm 0.005 } $ &               $  \mathbf{ 0.9873 \pm 0.0048 } $ \\
       & ind-RNN &                           $ 0.9995 \pm 0.00024 $ &                            $ 0.975 \pm 0.0053 $ &                            $ 0.9766 \pm 0.005 $ \\
       & ind-RNNSig &                           $ 0.9975 \pm 0.00078 $ &                            $ 0.9309 \pm 0.011 $ &                           $ 0.9364 \pm 0.0095 $ \\
       & ind-Sig &                              $ 0.6692 \pm 0.18 $ &                             $ 0.1848 \pm 0.19 $ &                             $ 0.1907 \pm 0.19 $ \\
       & lin-DeepSig &  $  \mathbf{ \underline{ 0.9998 \pm 6.8e-05 }} $ &   $  \mathbf{ \underline{ 0.989 \pm 0.0021 }} $ &     $  \mathbf{ \underline{ 0.99 \pm 0.002 }} $ \\
       & lin-RNN &                           $ 0.9993 \pm 0.00043 $ &                            $ 0.9767 \pm 0.009 $ &                            $ 0.9779 \pm 0.008 $ \\
       & lin-RNNSig &                           $ 0.9977 \pm 0.00037 $ &                           $ 0.9275 \pm 0.0066 $ &                           $ 0.9327 \pm 0.0066 $ \\
       & lin-Sig &                             $ 0.5502 \pm 0.067 $ &                           $ 0.09436 \pm 0.033 $ &                           $ 0.09958 \pm 0.041 $ \\
       & zero-DeepSig &               $  \mathbf{ 0.9998 \pm 0.00013 } $ &            $  \underline{ 0.9834 \pm 0.0064 } $ &            $  \underline{ 0.9845 \pm 0.0062 } $ \\
       & zero-RNN &                           $ 0.9989 \pm 0.00052 $ &                              $ 0.96 \pm 0.011 $ &                             $ 0.9625 \pm 0.01 $ \\
       & zero-RNNSig &                           $ 0.9968 \pm 0.00063 $ &                           $ 0.9215 \pm 0.0088 $ &                           $ 0.9274 \pm 0.0082 $ \\
       & zero-Sig &                              $ 0.97 \pm 0.0039 $ &                            $ 0.6996 \pm 0.042 $ &                              $ 0.717 \pm 0.04 $ \\
\bottomrule
\end{tabular}

% 	\label{supp: tab character}
% 	}
% 	\end{center}
% \end{table}

\begin{table}[htbp]
    \begin{center}
	\caption{Physionet 2012 }
	%\small{
	\begin{tabular}{lll}
\toprule
metric &                                          AUROC &                              Average Precision \\
model          &                                                &                                                \\
\midrule
GP-PoM-DeepSig &                           $ 82.084 \pm 0.836 $ &                           $ 47.858 \pm 1.362 $ \\
GP-PoM-RNN     &                           $ 83.222 \pm 0.570 $ &                           $ 49.263 \pm 1.115 $ \\
GP-PoM-RNNSig  &                           $ 77.879 \pm 1.072 $ &                           $ 40.157 \pm 0.539 $ \\
GP-PoM-Sig     &                           $ 74.388 \pm 2.211 $ &                           $ 33.909 \pm 2.902 $ \\
GP-DeepSig     &                           $ 82.164 \pm 0.245 $ &                           $ 47.707 \pm 0.971 $ \\
GP-RNN         &                           $ 81.196 \pm 0.953 $ &                           $ 47.322 \pm 1.642 $ \\
GP-RNNSig      &                           $ 74.665 \pm 1.763 $ &                           $ 36.585 \pm 1.549 $ \\
GP-Sig         &                           $ 70.984 \pm 1.611 $ &                           $ 30.886 \pm 2.115 $ \\
causal-DeepSig &                           $ 83.487 \pm 0.574 $ &                           $ 48.924 \pm 0.935 $ \\
causal-RNN     &            $  \underline{ 84.689 \pm 0.325 } $ &  $  \mathbf{ \underline{ 52.646 \pm 0.460 }} $ \\
causal-RNNSig  &                           $ 82.074 \pm 0.080 $ &                           $ 47.691 \pm 0.214 $ \\
causal-Sig     &                           $ 83.499 \pm 0.597 $ &                           $ 47.809 \pm 0.783 $ \\
ff-DeepSig     &                           $ 82.766 \pm 0.646 $ &                           $ 47.971 \pm 1.562 $ \\
ff-RNN         &  $  \mathbf{ \underline{ 84.954 \pm 0.157 }} $ &               $  \mathbf{ 52.427 \pm 0.521 } $ \\
ff-RNNSig      &                           $ 80.916 \pm 0.607 $ &                           $ 44.979 \pm 1.347 $ \\
ff-Sig         &                           $ 84.328 \pm 0.213 $ &                           $ 51.043 \pm 0.547 $ \\
ind-DeepSig    &                           $ 82.332 \pm 0.467 $ &                           $ 47.150 \pm 1.020 $ \\
ind-RNN        &               $  \mathbf{ 84.906 \pm 0.211 } $ &            $  \underline{ 51.887 \pm 0.713 } $ \\
ind-RNNSig     &                           $ 83.651 \pm 0.199 $ &                           $ 49.872 \pm 0.570 $ \\
ind-Sig        &                           $ 81.570 \pm 0.473 $ &                           $ 45.807 \pm 1.318 $ \\
lin-DeepSig    &                           $ 83.168 \pm 0.650 $ &                           $ 48.937 \pm 1.291 $ \\
lin-RNN        &                           $ 84.367 \pm 0.176 $ &                           $ 50.431 \pm 0.340 $ \\
lin-RNNSig     &                           $ 77.336 \pm 0.469 $ &                           $ 41.888 \pm 0.583 $ \\
lin-Sig        &                           $ 83.354 \pm 0.223 $ &                           $ 48.900 \pm 0.480 $ \\
zero-DeepSig   &                           $ 80.101 \pm 2.126 $ &                           $ 44.340 \pm 3.076 $ \\
zero-RNN       &                           $ 83.571 \pm 0.227 $ &                           $ 49.304 \pm 0.597 $ \\
zero-RNNSig    &                           $ 76.246 \pm 1.436 $ &                           $ 39.502 \pm 2.232 $ \\
zero-Sig       &                           $ 80.645 \pm 0.097 $ &                           $ 44.728 \pm 0.264 $ \\
\bottomrule
\end{tabular}

	\label{supp: tab physionet}
	%}
	\end{center}
\end{table}


\begin{figure}[tbp]
\subcaptionbox{CharacterTrajectories-R}{%
    \includegraphics[width=0.48\linewidth]{plots/barplot_main_CharacterTrajectories-MissingAtRandomSubsampler.pdf}\quad%
     \includegraphics[width=0.48\linewidth]{plots/barplot_main_params_CharacterTrajectories-MissingAtRandomSubsampler.pdf}%
   }\\%
  \subcaptionbox{CharacterTrajectories-L}{%
    \includegraphics[width=0.48\linewidth]{plots/barplot_main_CharacterTrajectories-LabelBasedSubsampler.pdf}\quad%
     \includegraphics[width=0.48\linewidth]{plots/barplot_main_params_CharacterTrajectories-LabelBasedSubsampler.pdf}%
  }
  \subcaptionbox{LSST-R}{%
    \includegraphics[width=0.48\linewidth]{plots/barplot_main_LSST-MissingAtRandomSubsampler.pdf}\quad%
    \includegraphics[width=0.48\linewidth]{plots/barplot_main_params_LSST-MissingAtRandomSubsampler.pdf}%
  }\\%
  \subcaptionbox{LSST-L}{%
    \includegraphics[width=0.48\linewidth]{plots/barplot_main_LSST-LabelBasedSubsampler.pdf}\quad%
    \includegraphics[width=0.48\linewidth]{plots/barplot_main_params_LSST-LabelBasedSubsampler.pdf}
  }%
  \caption{%
    Visualisations for \texttt{CharacterTrajectories} and \texttt{LSST} . The rows indicate datasets and different subsampling schemes (R for Random, L for Label-based). The left column displays the performance metric which was optimzied for: balanced accuracy (BAC), or average precision. The right column indicates the number of trainable parameters which the best model required (as selected in the hyperparameter search).
  }
  \label{supp: barplots1}
\end{figure}
\begin{figure}
   \subcaptionbox{PenDigits-R}{%
     \includegraphics[width=0.48\linewidth]{plots/barplot_main_PenDigits-MissingAtRandomSubsampler.pdf}\quad%
     \includegraphics[width=0.48\linewidth]{plots/barplot_main_params_PenDigits-MissingAtRandomSubsampler.pdf}
   }\\%
  \subcaptionbox{PenDigits-L}{%
    \includegraphics[width=0.48\linewidth]{plots/barplot_main_PenDigits-LabelBasedSubsampler.pdf}\quad%
    \includegraphics[width=0.48\linewidth]{plots/barplot_main_params_PenDigits-LabelBasedSubsampler.pdf}
  }\\ %
  \subcaptionbox{Physionet2012}{%
    \includegraphics[width=0.48\linewidth]{plots/barplot_main_Physionet2012-.pdf}\quad%
    \includegraphics[width=0.48\linewidth]{plots/barplot_main_params_Physionet2012-.pdf}
  }
  \caption{%
    Visualisations for \texttt{PenDigits} and \texttt{Physionet} . The rows indicate datasets and different subsampling schemes (R for Random, L for Label-based). The left column displays the performance metric which was optimzied for: balanced accuracy (BAC), or average precision. The right column indicates the number of trainable parameters which the best model required (as selected in the hyperparameter search).
  }
  \label{supp: barplots2}
\end{figure}

\begin{figure}[tbp] 
    \begin{center}
    \includegraphics[width=0.95\linewidth]{plots/gp_training_plot.pdf}\quad%
    % \includegraphics[width=0.48\linewidth]{plots/barplot_main_params_CharacterTrajectories-LabelBasedSubsampler.pdf}%
  \end{center}
  \caption{GP-PoM training illustrated for CharacterTrajectories as compared to conventional GP adapter.}
  \label{supp: gp-training}
\end{figure}



%%%%%%%%%%%%%%%%%%%%%%%%%%%%%%%%%%%%%%%%%%%%%%%%%%%%%%%%%%%%%%%%%%%%%%%%
\subsection{Imputation strategies} \label{supp: Imputation}
%%%%%%%%%%%%%%%%%%%%%%%%%%%%%%%%%%%%%%%%%%%%%%%%%%%%%%%%%%%%%%%%%%%%%%%%

We consider the following set of strategies for path imputation, i.e.\
\begin{enumerate}
    \item linear interpolation: At a given imputation point, the previous and next observed data point are linearly interpolated.
    \item forward filling: At a given imputation point, the last observed value is carried forward.
    \item indicator imputation: At a given imputation point, for each feature dimension, if no observation is available an binary missingness indicator variable is set to $1$, $0$ otherwise. The missing value is filled with $0$.
    \item zero imputation: At a given imputation point, missing values are filled with $0$.
    \item causal imputation: This approach is related to forward filling
      and motivated by signature theory. As opposed to forward filling,
      the time and the actual value are updated sequentially. For more
      details, we introduce causal imputation in Section~\ref{sec:Causal signature imputation}.
    \item Gaussian process adapter: We introduce GP adapters in
      Section~\ref{sec: GPadapter}, where $\mathbf{z}$
      refers to the imputed time series~(modelled as Gaussian
      distribution).
\end{enumerate}


%%%%%%%%%%%%%%%%%%%%%%%%%%%%%%%%%%%%%%%%%%%%%%%%%%%%%%%%%%%%%%%%%%%%%%%%
\subsection{Dataset statistics and filtering} \label{supp: Dataset stats}
%%%%%%%%%%%%%%%%%%%%%%%%%%%%%%%%%%%%%%%%%%%%%%%%%%%%%%%%%%%%%%%%%%%%%%%%
\paragraph{Physionet2012}
As our focus is time series classification, for \texttt{Physionet2012}~\citep{goldberger2000physiobank}, we included the 36 time series variables, and excluded the static covariates (notably, we counted the variable `weight' as a static covariate). Subsequently, we excluded the following $12$ icu stays (here represented by there ids) for having no time series data (but only static covariates): $140501, 150649, 140936, 143656, 141264, 145611, 142998, 147514,$  $142731, 150309, 155655, 156254 $, and a single noisy encounter, $135365$, which contained much more observations than all other patients. After these filtering steps, we count $11987$ instances and a binary class label, whether a patient survives the hospital stay or not.

\paragraph{PenDigits}
For \texttt{PenDigits}~\citep{Dua2019}, we count $10992$ samples, featuring $2$ channels and $8$ time steps, and $10$ classes.

\paragraph{LSST}
LSST~\citep{allam2018photometric} contains $4925$ instances featuring $6$ channel dimensions and $36$ time steps. This dataset contains $14$ classes.

 \paragraph{CharacterTrajectories~\citep{Dua2019}}
 This dataset contains $2858$ instances, featuring $3$ channel dimensions, $182$ time steps and $20$ classes.


%%%%%%%%%%%%%%%%%%%%%%%%%%%%%%%%%%%%%%%%%%%%%%%%%%%%%%%%%%%%%%%%%%%%%%%%
\subsection{Model implementations, architectures and hyperparameters} \label{supp: Model Architectures}
%%%%%%%%%%%%%%%%%%%%%%%%%%%%%%%%%%%%%%%%%%%%%%%%%%%%%%%%%%%%%%%%%%%%%%%%

All models are implemented in Pytorch~\citep{pytorch2019}, whereas the
GP adapter and \textsc{GP-PoM} are implemented using the GPyTorch
framework~\citep{gardner2018gpytorch}. Next, we specify the details of
the model architectures.

\paragraph{\textsc{Sig}}
We use a simple signature model that involves one        signature block comprising of a linear
    augmentation followed by the signature transform. Subsequently, a final module of dense layers $(30,30)$ is used. This is architecture refers to the Neural-signature-augment model \cite{kidger2019deep}.
  %
  \paragraph{\textsc{RNNSig}} This model extends the signature transform to a window-based
    stream of signatures, where the final neural module is a GRU
    sliding over the stream of signatures. We allowed window sizes between $3$ and $10$ steps. For the GRU cell, we allowed any of the following number of hidden units: $[16,32,64,128]$.
    
  \paragraph{\textsc{RNN}} Here, we use a standard RNN model using GRU cells. The size of hidden units was chosen as one of the following: $[16,32,64,128, 256, 512]$.
  %
  \paragraph{\textsc{DeepSig}} For the deep signature model we  employ two signature blocks (each comprising a linear augmentation and the signature calculation) following \citet{kidger2019deep}.

\subsubsection{Hyperparameters}

For all signature-based models, we allowed a signature truncation
depth of $2$--$4$, as we observed that larger values quickly led to
a parameter explosion. All models were optimized using Adam
\citep{kingma2014adam}. Both the learning rates and weight decay
were drawn log-uniformly between $10^{-4}$ and $10^{-2}$. We allowed
for the following batch-sizes: $(32, 64, 128, 256)$. For GP-based
models, to save memory, we used virtual batching based on
a batch-size of~$32$. 
    

%%%%%%%%%%%%%%%%%%%%%%%%%%%%%%%%%%%%%%%%%%%%%%%%%%%%%%%%%%%%%%%%%%%%%%%%
\subsection{Fragile dependence on sampling in unrelated channels: example}
\label{sec:Fragile dependence} 
%%%%%%%%%%%%%%%%%%%%%%%%%%%%%%%%%%%%%%%%%%%%%%%%%%%%%%%%%%%%%%%%%%%%%%%%

%%%%%%%%%%%%%%%%%%%%%%%%%%%%%%%%%%%%%%%%%%%%%%%%%%%%%%%%%%%%%%%%%%%%%%%%
\begin{figure}[t]
    \centering
    \includegraphics[width=0.35 \columnwidth]{figures/sig_path3.pdf}
    \caption{
      L{\'e}vy area of the forward-fill imputed path. By changing
      $t_{3/2}$~(a \emph{single} unrelated observation!), we can make
      this disparity greater or smaller.
    }
    \label{fig:bentline}
\end{figure}
%%%%%%%%%%%%%%%%%%%%%%%%%%%%%%%%%%%%%%%%%%%%%%%%%%%%%%%%%%%%%%%%%%%%%%%%

Suppose that we have observed the~(very short) time series
%
\begin{equation}\label{eq:flaw1}
    \mathbf{x} = ((t_1, x_1^1, x_1^2), (t_2, x_2^1, *)) \in \seriesspace(\reals^2).
\end{equation}
%
Perhaps we now apply, say, forward fill data-imputation, to produce
%
\begin{equation*}
    ((t_1, x_1^1, x_1^2), (t_2, x_2^1, x_1^2)).
\end{equation*}
%
Finally we linearly path-impute to create the linear path
%
\begin{align*}
    f &\colon [t_1, t_2] \to \reals \times \reals^2\\
    f &\colon t \mapsto \left(t, x_1^1\frac{t_2 - t}{t_2 - t_1} + x_2^1\frac{t - t_1}{t_2 - t_1}, x_1^2\right),
\end{align*}
%
to which we may then apply the signature transform. In particular we
will have computed the L{\'e}vy area with respect to $t$ and $x^1$. As this
is just a straight line, the L{\'e}vy area is zero.

Now suppose we include an additional observation at some time $t_{3/2} \in (t_1, t_2)$, so that our data is instead
%
\begin{equation}\label{eq:flaw2}
    \mathbf{x} = ((t_1, x_1^1, x_1^2), (t_{3/2}, *, x_{3/2}^2), (t_2, x_2^1, *)).
\end{equation}
%
Then the same procedure as before will produce the data
%
\begin{equation*}
    \mathbf{x} = ((t_1, x_1^1, x_1^2), (t_{3/2}, x_1^1, x_{3/2}^2), (t_2, x_2^1, x_{3/2}^2)),
\end{equation*}
%
with corresponding function $f$. The $(t, x^1)$ components of $f$ and
its $(t, x^1)$-L{\'e}vy area are shown in Figure~\ref{fig:bentline}. As
a result of an unrelated observation in the $x^2$ channel, the $(t,
x^1)$-L{\'e}vy area has been changed.
%
The closer $t_{3/2}$ is to $t_2$, the greater the disparity.
%
This simple example underscores the danger of `just forward-fill
data-imputing'. Doing so has introduced an undesired dependency on the
simple \emph{presence} of an observation in other channels, with the
change in our imputed path being determined by the \emph{time} at which
this other observation occurred.

Indeed, \emph{any} imputation scheme that predicts something other than
the unique value lying on the dashed line in Figure~\ref{fig:bentline},
will fail. This means that this example holds for essentially every
data-imputation scheme---the only scheme that survives this flaw is the
linear data-imputation scheme. This is the unique imputation scheme
that coincides with the linear path-imputation that \emph{must} be our
concluding step.
%
However, when there is missing data at the start or the end of
a partially observed times series, then there is no `next observation'
which linear imputation may use. So in general, we cannot uniformly
apply the linear data-imputation scheme, and must choose another scheme.

%%%%%%%%%%%%%%%%%%%%%%%%%%%%%%%%%%%%%%%%%%%%%%%%%%%%%%%%%%%%%%%%%%%%%%%%
\subsection{Causal signature imputation}\label{sec:Causal signature imputation}
%%%%%%%%%%%%%%%%%%%%%%%%%%%%%%%%%%%%%%%%%%%%%%%%%%%%%%%%%%%%%%%%%%%%%%%%

In Section \ref{sec:Fragile dependence} we have spoken about the limitations of traditional data-imputation schemes, and at first glance one may be forgiven for
thinking that these are issues are unavoidable. 
%
However, it turns out that we need not be limited just to these
traditional imputation schemes. The trick is to consider time not as
a \emph{parameterisation}, but as a \emph{channel}\footnote{To be clear, using time as a channel is already
a well-known trick in the signature literature that we do not take
credit for inventing! See for example \citet[Definition
A.3]{kidger2019deep}. It is however pleasing that something commonly
used in the theory of signatures is also what allows us to overcome what
we identify as some of their limitations.}.
%
This leads to a `meta imputation strategy', which we refer to as
\emph{causal signature imputation}. It will turn any traditional causal
data-imputation strategy (for example, feed-forward) into a causal
path-imputation strategy for signatures; at the same time it will
overcome the issue of a fragile dependence.

Suppose we have $\mathbf{x} \in \seriesspace(\dataspace^*)$, and some
favourite choice of causal data-imputation strategy $c \colon
\seriesspace(\dataspace^*) \to \seriesspace(\dataspace)$.
%
Next, given
%
\begin{align}
    \mathbf{x} = ((t_1, x_1), \ldots, (t_n, x_n)) \in \seriesspace(\dataspace),
\end{align}
%
we define the operation $\Omega \colon \seriesspace(\dataspace) \to \seriesspace(\dataspace)$ by
%
\begin{align}
    \Omega(\mathbf{x}) = (&(t_1, x_1), (t_2, x_1), (t_2, x_2),(t_3, x_2),\nonumber\\
    &\ldots,\nonumber\\
    &(t_i, x_i), (t_{i + 1}, x_i), (t_{i + 1}, x_{i + 1}), (t_{i + 2}, x_{i + 1}),\nonumber\\
    &\ldots,\nonumber\\
    &(t_{n - 1}, x_{n - 1}), (t_n, x_{n - 1}),(t_n, x_n)).\label{eq:causalsig}
\end{align}
%
That is, \emph{first} time is updated, and \emph{then} the corresponding observation
in data space is updated. This means that the change in data space
occurs instantaneously.

For each $n \in \naturals$~(and given $a < b$), fix any $s_i^{(n)}$ for
$i \in \{1, \ldots, n \}$. 
(We will see that the exact choice is unimportant in a moment.)
%
Given
%
\begin{align*}
    \mathbf{x} = ((t_1, x_1), \ldots, (t_n, x_n)) \in \seriesspace(\dataspace),
\end{align*}
%
let $\psi \colon \seriesspace(\dataspace) \to (\reals \times
\dataspace)^{[a, b]}$ be the unique continuous piecewise linear path
such that $\psi(s_i^{(n)}) = (t_i, x_i)$. Note that this is just
a slight generalisation of the linear path-imputation that has already
been performed so far; we are simply no longer asking for additional
assumptions of the form $s_i^{(n)} = t_i$.\footnote{As in the
$\mathbf{\varphi}_\theta$ of \cite{toth2019gp}, for example.}

Finally, we put this all together, and define the causal signature imputation strategy $\phi_c$ associated with $c$ to be
%
\begin{equation*}
\phi_c = \psi \circ \Omega \circ c,
\end{equation*}
which will be a map $\seriesspace(\dataspace^*) \to (\reals \times \dataspace)^{[a, b]}$.
%
Thus $\phi_c$ defines a family of path-imputation schemes, parameterised by a choice of data-imputation scheme.

Before we analyse \emph{why} this works in practice, we repeat a crucial
property of the signature transform~\citep[Appendix~A]{kidger2019deep}.
%
\begin{theorem}[Invariance to reparameterisation]\label{theorem:invariancetime}
  Let $f \colon [a, b] \to \reals^d$ be a continuous piecewise
  differentiable path. Let $\psi \colon [a, b] \to [c, d]$ be continuously
  differentiable, increasing, and surjective. Then $\sig(f) = \sig(f \circ
  \psi)$.
\end{theorem}
%
Coming back to our analysis, we first note that the previous theorem
implies that the signature transform of $\phi_c(\mathbf{x})$ is
invariant to the choice of $s_i^{(n)}$.
%
Second, note that holding time between observations fixed is a valid
choice, by the definition for $\seriesspace$ in equation
\eqref{eq:seriesspace}. There should hopefully be no moral objection to
our definition of $\seriesspace$, as holding time fixed essentially just
corresponds to a jump discontinuity; not such a strange thing to have
occur. Here, by replacing time as the parameterisation, we are then able
to recover the continuity of the path.
%
Third, we claim that  $\phi_c$ is immune to the two major flaws of
imputation methods, namely
%
\begin{inparaenum}[(i)]
  \item their fragile dependence on sampling in unrelated channels, and
  \item their non-causality.
\end{inparaenum}
%
Let us consider the first flaw of dependence on sampling in unrelated channels.
%
For simplicity, take $c$ to be the forward-fill data-imputation
strategy. Consider again the $\mathbf{x}$ defined in expression~\eqref{eq:flaw1}.
This means that
%
\begin{equation}\label{eq:causal1}
    \phi_c(\mathbf{x}) = \psi(\;((t_1, x_1^1, x_1^2), (t_2, x_1^1, x_1^2), (t_2, x_2^1, x_1^2))\;).
\end{equation}
%
Contrast adding in the extra observation at $t_{3/2}$ as in equation \eqref{eq:flaw2}. Then
%
\begin{align}
    &\phi_c(\mathbf{x})(s)\nonumber\\
    &=\psi(\;((t_1, x_1^1, x_1^2), (t_{3/2}, x_1^1, x_1^2), (t_{3/2}, x_1^1, x_{3/2}^2),\nonumber\\ &\hspace{3.1em}(t_2, x_1^1, x_{3/2}^2), (t_2, x_2^1, x_{3/2}^2))\;).\label{eq:causal2}
\end{align}
%
Evaluating each $\psi$ will then in each case give a path with three
channels, corresponding to $t, x^1, x^2$. Then it is clear that the $(t,
x^1)$ component of the path in equation~\eqref{eq:causal1} is just
a reparameterisation of the path in equation~\eqref{eq:causal2},
a difference which is irrelevant by Theorem~\ref{theorem:invariancetime}.
(And the $x^2$ component of the second
path has been updated to use the new information $x_{3/2}^2$.) Thus the causal path impuation scheme is robust to such issues. For general time series and
$c$ taken to be any other causal data-imputation strategy, then much the
same analysis can be easily be performed.

Now consider the second potential flaw, of non-causality. The issue
previously arose because of the non-causality of the linear
path-imputation. We see from equation \eqref{eq:causalsig}, however,
such changes only occur in data space while the time channel is frozen;
conversely the time channel only updates with the value in the data
space frozen. Provided that $c$ is also causal, then causality will,
overall, have been preserved. For example, it is possible to use this
scheme in an online setting.
%
There are interesting comparisons to be made between causal signature
imputation and certain operations in the signature literature. First is
the \emph{lead-lag} transform \cite{primer2016}. With the lead-lag
transform, the entire path is \emph{duplicated}, and then each side is
alternately updated. Conversely, in causal signature imputation, the
path is instead \emph{split} between $t$ and $(x^1, \ldots, x^n)$, and
then each side is alternately updated.
%
Second is the comparison to the linear and rectilinear embedding
strategies, see for example \cite{fermanian2019embedding}. It is
possible to interpret $\psi \circ \Phi$ as a hybrid between the linear
and rectilinear embeddings: it is rectilinear with respect to an
ordering of $t$ and $(x^1, \ldots, x^n)$, and linear on $(x^1, \ldots,
x^n)$.
Furthermore, the time-joined transformation \cite{levin2013} is pursuing a very similar goal to the here described causal signature imputation. This is also why we do not consider this imputation strategy as a novel contribution of this work.

\end{document}



    

